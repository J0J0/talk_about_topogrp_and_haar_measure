\chapter{Existenz und Eindeutigkeit des Haar'schen Maßes} \label{chap:pf}
Wir wollen nun die Existenz und Eindeutigkeit (bis auf einen positiven Faktor)
eines Haar-Maßes auf LKH-Gruppen zeigen. 
Für den Existenzbeweis halten wir uns an den erstmals 1940 von 
\emph{Andr\'e Weil} in allgemeiner Form veröffentlichten Beweis für beliebige
LKH-Gruppen, wie man ihn etwa bei
Elstrod\cite[Kap.\,VIII,\;\S3,\;3.12]{bookc:elstrod11} 
oder auch bei Folland\cite[Ch.\,11,\;\S11.8]{bookc:folland99} findet.
Dieser benutzt das \emph{Auswahlaxiom} in Form des
\emph{Satzes von Tychonoff}, weshalb es an dieser Stelle erwähnenswert ist, dass
\emph{Henri Cartan} (ebenfalls 1940) auch einen Beweis gefunden hat, der ohne
selbiges auskommt. (Siehe dazu auch: Alfsen\cite{artcle:alfsen63}.)
Für den Eindeutigkeitsbeweis verwenden wir ein Argument, das hauptsächlich auf
dem Gebrauch des \emph{Satzes von Fubini} beruht, siehe beispielsweise
Folland\cite[Ch.\,11,\;\S11.9]{bookc:folland99}. Auch hier gibt es alternative
Beweise, die mit elementareren Mitteln auskommen. Der ursprüngliche Beweis von
\emph{A. Weil} macht zum Beispiel keinen Gebrauch vom Satz von Fubini und ein
Beweis nach diesem Schema findet man im oben schon zitierten Satz bei Elstrod.

\medskip
Da wir schon gezeigt haben, dass sich Haar-Maße und Haar-Integrale entsprechen,
genügt es, die Existenz und (im Wesentlichen) Eindeutigkeit eines Haar-Integrals
zu zeigen, und dies ist auch der übliche Ansatz. Im Übrigen meinen wir hier und
im Rest dieses Kapitels stets \emph{linkes} Haar-Maß und \emph{linkes}
Haar-Integral, wenn wir es nicht explizit spezifizieren. Um nun eine Motivation
für die weiter unten auftrende Konstruktion zu geben, betrachten wir folgendes
Szenario: Sei $G$ eine LKH-Gruppe und seien $K\subset G$ eine kompakte und
$U\in\frakU$ eine offene Menge. Weil wir ein von innen reguläres Maß suchen,
genügt es prinzipiell, das Maß auf kompakten Mengen zu kennen. Wir möchten nun
also ein \enquote{Maß} für $K$ angeben, was wir zum Beispiel unter Verwendung
von $U$ wie folgt tun können:
Weil $K$ kompakt ist, finden wir ein $n\in\N$ und endlich viele $x_i\in G$, so
dass $(x_iU)_{i\in\{1,\ldots,n\}}$ eine Überdeckung von~$K$ bildet. Definiert
man nun $(K : U)$ als das minimale derartige $n\in\N$, so misst $(K : U)$ in
gewissem Rahmen wie groß $K$ in Relation zu $U$ ist. Die Idee ist nun, $U$ immer
kleiner zu wählen, so dass $U$ auf $\{e\}$ zusammenschrumpft und man somit die
Menge $K$ mit immer weniger \enquote{Überlappungen} überdecken kann. Mittels
einer Normierung, also $(K : U)/(K_0 : U)$ für festes $K_0\subset G$, können wir
erhoffen, dass sich bei dem eben skizzierten Grenzprozess ein Wert für diesen
Quotienten einstellt und wir dadurch ein (offenbar linksinvariantes) Maß auf~$G$
bekommen.

Wie allerdings gerade schon angesprochen, ist es einfacher, die Sichtweise auf
Funktionale zu verlagern und in ähnlicher Herangehensweise ein Haar-Integral auf
einer beliebigen (aber festen) LKH-Gruppe zu konstruieren. Im Detail
funktioniert dies so:


\section{Existenz eines Haar-Maßes}
Sei ab jetzt $G$ eine fixierte LKH-Gruppe. Sind dann $f,\phi\in\contcompplus$,
so ist 
\[ V \defeq \bigl\{ x \in G \Mid \phi(x) > \thalf\supnorm\phi \bigr\} 
      = \phi^{-1}\Bigl( \bigl( \thalf\supnorm\phi, \mkern2mu\infty\bigr) \Bigr)
      \neq \emptyset
\]
eine offene Menge und weil $\supp(f)$ kompakt ist, gibt es ein $n\in\N$ und 
endlich viele $x_i\in G$, so dass $\supp(f)$ von $(x_iV)_{i\in\{1,\ldots,n\}}$ 
überdeckt wird. Dann gilt für $x\in\supp(f)$:
\[
    \biggl( \frac{2}{\supnorm\phi} \sum_{i=1}^n L_{x_i}\phi \biggr)(x)
    = \frac{2}{\supnorm\phi} \sum_{i=1}^n \phi(x_i^{-1}x)
\]
Wegen der Wahl von $x$ gibt es ein $x_k$ (mit $k\in\N$), so
dass $x\in x_kV$ bzw. $x_k^{-1}x\in V$ erfüllt ist. Daraus folgt:
\[  
    \frac{2}{\supnorm\phi} \sum_{i=1}^n \phi(x_i^{-1}x)
    \geq \frac{2}{\supnorm\phi} \, \phi(x_k^{-1}x)
    > \frac{2}{\supnorm\phi} \, \frac{\supnorm\phi}{2} 
    = 1 \geq \frac{f(x)}{\supnorm f}
\]
Trivialerweise ist die Ungleichung auch für $x\notin\supp(f)$ wahr, so dass
gilt:
\[  f \leq \frac{2\supnorm f}{\supnorm\phi} \sum_{i=1}^n L_{x_i}\phi  \]
Diese Idee nutzen wir nun zu einer allgemeineren Definition, bei der wir
unterschiedliche Koeffizienten für jeden Summanden zulassen:

\begin{thDef}[(Haar-)Überdeckungszahl]
    \label{pf:def:covernum}
    %
    Für $f,\phi \in\contcompplus$ definiert
    \[
        (f : \phi) \defeq 
        \inf \biggl\{\, \sum_{i=1}^n c_i  \cMid[\;]{\bigg} 
                     f \leq \sum_{i=1}^n c_i \, L_{x_i}\phi 
                     \qtextq{mit} n\in\N, \;\;
                     \forall i\in\{1,\ldots,n\}\colon 
                     c_i\in\R[>0] \wedge x_i\in G
            \biggr\}
    \]
    die sogenannte \emph{(Haar-)Überdeckungszahl} von $f$ und $\phi$.
\end{thDef}

Nach der Vorüberlegung ist die rechte Menge nicht leer und offenbar ist sie
durch~$0$ von unten beschränkt; somit ist die Überdeckungszahl wohldefiniert.
Weiter gelten folgende mehr oder weniger offensichtliche Aussagen:
%
\begin{thLemma}[Eigenschaften der Überdeckungszahl]
    \label{pf:covernumprops}
    %
    Für $f,g,\phi \in\contcompplus$ und $x\in G$ sowie $\lambda\in\R[>0]$ gilt:
    \begin{enumerate}[a)]\vspace{-4pt}
        \item
            Abschätzung nach unten:\quad
            $0 < \supnorm{f}/\supnorm{\phi} \leq (f : \phi)$
        \item
            Linksinvarianz im ersten Argument:\quad
            $(L_xf : \phi) = (f : \phi)$.
        \item
            Verträglichkeit mit Skalarmultiplikation:\quad
            $(\lambda f : \phi) = \lambda (f : \phi)$.
        \item
            Subadditivität:\quad
            $(f\mkern1mu{+}\mkern2mu g : \phi) \leq (f : \phi) + (g : \phi)$.
        \item\label{pf:covernumprops:reducing}
            \enquote{Abgeschätztes Kürzen}:\quad
            $(f : g)\,(g : \phi) \geq (f : \phi)$.
    \end{enumerate}
\end{thLemma}

\begin{proof}
    Seien alle Variablen wie in der Behauptung gewählt und seien weiter
    \[  f \leq \sum_{i=1}^n c_i \, L_{x_i}\phi
        \qqundqq
        g \leq \sum_{j=1}^m d_j \, L_{y_j}\phi
    \]
    (mit $c_i,d_j\in\R[>0],\; x_i,y_j\in G$ für alle $i,j$) 
    Ausdrücke wie in der Definition der Überdeckungszahl \pcref{pf:def:covernum}.
    \begin{enumerate}[(a)]
        \item 
            Bilden wir auf beiden Seiten der Ungleichung für $f$ die
            Supremumsnorm, so erhalten wir mit der Dreiecksungleichung:
            \[ \supnorm{f} \leq \sum_{i=1}^n c_i \,
                \underbrace{\supnorm{L_{x_i}\phi}}_{=\supnorm\phi}
            \]
            Durch Umstellen und Infimumsbildung folgt die Behauptung.
            
        \item
            Es gilt
            \[  f \leq \sum_{i=1}^n c_i \, L_{x_i}\phi 
                \qiffq
                L_xf \leq L_x \biggl( \sum_{i=1}^n c_i \, L_{x_i}\phi \biggr)
                        = \sum_{i=1}^n c_i \, L_{xx_i}\phi
            , \]
            woraus sofort die Behauptung folgt.
            
        \item
            Hier gilt analog:
            \[  f \leq \sum_{i=1}^n c_i \, L_{x_i}\phi 
                \qiffq
                \lambda f \leq \lambda \sum_{i=1}^n c_i \, L_{x_i}\phi
            \]
            
        \item
            Addition der Ungleichungen für $f$ und $g$ liefert:
            \[ f+g \leq 
                \sum_{i=1}^n c_i\,L_{x_i}\phi + \sum_{j=1}^m d_j\,L_{y_j}\phi
            \]
            Alle rechten Kombinationen von $c_i,x_i,d_j,y_j$ können offenbar
            auch mit der Wahl entsprechender $\gamma_k\in\R[\geq0],\; z_k\in G$
            in folgender Formel erreicht werden:
            \[ f+g \leq \sum_{k=1}^N \gamma_k \, L_{z_k}\phi \]
            Dies sind aber genau die Ausdrücke aus \cref{pf:def:covernum},
            welche wir zur Bestimmtung von $(f\mkern1mu{+}\mkern2mu g : \phi)$
            beachten müssen. Daraus folgt zunächst
            \[ (f\mkern1mu{+}\mkern2mu g : \phi) \leq 
                    \sum_{i=1}^n c_i + \sum_{j=1}^m d_j
            \]
            und durch Infimumsbildung wieder die Behauptung.
            
        \item
            Wir betrachten die Ungleichung für $f$, aber mit $g$ statt $\phi$, und
            benutzen dann die Ungleichung für $g$ und $\phi$, um folgendermaßen
            abzuschätzen:
            \[ f \leq \sum_{i=1}^n c_i \, L_{x_i}g 
                \leq \sum_{i=1}^n c_i \, L_{x_i}\biggl( 
                        \sum_{j=1}^m d_j \, L_{y_j}\phi
                    \biggr)
                = \sum_{i=1}^n\sum_{j=1}^m c_i d_j \, L_{x_iy_j}\phi
            \]
            Wegen
            \[ \sum_{i=1}^n\sum_{j=1}^m c_i d_j 
                =   \biggl(\mkern2mu \sum_{i=1}^n c_i \biggr) \, 
                    \biggl(\mkern2mu \sum_{j=1}^m d_j \biggr)
            \]
            folgt die Behauptung erneut durch Infimumsbildung.
    \end{enumerate}
\end{proof}

\medskip
Nun wählen wir ein $f_0\in\contcompplus$ beliebig, aber fest für den Rest des
Abschnitts. Dieses $f_0$ wird (wie $K_0$ in der Einleitung) die Rolle der
Normierung übernehmen, wie wir später sehen werden.

\begin{thDef}[Approximierende Funktionale]
    Zu jedem $\phi \in\contcompplus$ definieren wir ein Funktional $I_\phi$ wie folgt:
    \begin{align*}
        I_\phi\colon \contcompplus &\to         \R[>0]      \\
                                f  &\mapsto     \frac{(f : \phi)}{(f_0 : \phi)}
    \end{align*}
\end{thDef}
%
Der Titel der Definition deutet schon darauf hin, als was wir diese Funktionale
auffassen wollen: als Annäherungen an das gesuchte Haar-Integral. Dabei bekommen
wir einen besseren Wert, je kleiner der Träger von $\phi\in\contcompplus$ wird
(analog zur Idee in der Einleitung, $U$ auf $\{e\}$ zusammenschrumpfen zu
lassen). Dass diese $I_\phi$ schon die meisten gewünschten Eigenschaften
besitzen, zeigt folgendes Resultat:

\begin{thLemma}[Eigenschaften der approximierenden Funktionale]
    \label{pf:Iphiprops}
    %
    Seien $f,g,\phi \in\contcompplus$ und seien $x\in G$ sowie $\lambda\in\R[>0]$.
    Dann gelten für $I_\phi$ folgende Aussagen:
    \begin{enumerate}[a)]
        \item\label{pf:Iphiprops:leftinvariance}
            Linksinvarianz:\quad
            $I_\phi(L_xf) = I_\phi(f)$.
        \item\label{pf:Iphiprops:scalarmult}
            Verträglichkeit mit Skalarmultiplikation:\quad
            $I_\phi(\lambda f) = \lambda \, I_\phi(f)$.
        \item\label{pf:Iphiprops:subadditive}
            Subadditivität:\quad
            $I_\phi(f+g) \leq I_\phi(f) + I_\phi(g)$.
        \item\label{pf:Iphiprops:range}
            Eingrenzung des Wertebereichs:\quad
            $I_\phi(f) \in \bigl[ (f_0 : f)^{-1}, \; (f : f_0) \bigr]$.
    \end{enumerate}
\end{thLemma}

\begin{proof}
    Die ersten drei Eigenschaften folgen unmittelbar aus den entsprechenden 
    Aussagen in \cref{pf:covernumprops}.
    \\
    Wir zeigen noch \ref{pf:Iphiprops:range}: Betrachte die Definition von
    $I_\phi$ und verwende \mycref{pf:covernumprops:reducing}, um folgende
    Abschätzungen einzusehen:

    \begin{align*}
        \frac{(f : \phi)}{(f_0 : \phi)} 
        &\leq \frac{(f : f_0)\,(f_0 : \phi)}{(f_0 : \phi)}
        = (f : f_0)
        \\
        \shortintertext{und}
        %
        \frac{(f : \phi)}{(f_0 : \phi)} 
        &\geq \frac{(f : \phi)}{(f_0 : f)\,(f : \phi)}
        = (f_0 : f)^{-1}
        .
    \end{align*}
\end{proof}

Der einzige Defekt, den die $I_\phi$ gegenüber dem gesuchten Haar-Integral noch
haben, ist die fehlende Additivität. \mycref{pf:Iphiprops:subadditive} zeigt
zwar schon die Subadditivität auf, jedoch ist nun der entscheidende Punkt im
Beweis das folgende Lemma, welches zeigt, dass auch die umgekehrte Abschätzung
für jeden gegebenen Fehler für klein genuge Träger von $\phi$ erfüllt ist.
Etwas präziser formuliert bedeutet dies:

\begin{thLemma}[Nahezu-Additivität gewisser approximierender Funktionale]
    \label{pf:approxadditive}
    %
    Seien $f_1,f_2 \in\contcompplus$ und sei $\epsilon\in\R[>0]$ beliebig.
    Dann gibt es eine Umgebung $U\in\frakU$ um $e$, so dass für alle
    $\phi\in\contcompplus$ mit $\supp(\phi)\subset U$ gilt:
    \[ I_\phi(f_1) + I_\phi(f_2) \leq I_\phi(f_1+f_2) + \epsilon  \]
\end{thLemma}

\begin{proof}
    Sei $g\in\contcompplus$ derart, dass $g\vert_{\supp(f_1+f_2)} \equiv 1$
    gilt. Die Existenz eines solchen $g$ wird durch eine Variante des
    \emph{Urysohn'schen Lemmas} für lokalkompakten Hausdorfräume gesichert,
    wie man sie zum Beispiel bei Folland\cite[Ch.\,4,\;\S4.32]{bookc:folland99} 
    % TODO: Referenz auf vorherigen Vortrag einfügen
    findet. Weiter sei $\delta\in\R[>0]$ so gewählt, dass
    \[ 2\delta\, (f_1\mkern1mu{+}\mkern1mu f_2 : f_0) 
        + \delta(1+2\delta) \, (g : f_0) \leq \epsilon
    \]
    gilt. Warum wir $\delta$ gerade so wählen, ist zu diesem Zeitpunkt nicht
    offensichtlich und wird erst am Ende des Beweises klar werden.
    Wir definieren nun $h,h_1,h_2$ wie folgt:
    \[ h \defeq f_1+f_2+\delta g \qqundqq
        h_j \defeq \frac{f_j}{h} \quad \text{für $j\in\{1,2\}$,}
    \]
    wobei wir $h_1,h_2$ außerhalb von $\supp(f_1)$ bzw. $\supp(f_2)$ identisch
    null setzen. Dann gilt offenbar $h_1,h_2\in\contcompplus$, weswegen beide
    Funktionen nach \cref{tg:unicont} rechts-gleichmäßig stetig sind und es somit
    Umgebungen $U_1,U_2\in\frakU$ von $e$ gibt, so dass für $j\in\{1,2\}$ und
    alle $y\in U_j$ gilt: $\supnorm{h_j - R_yh_j} < \delta$. Setze dann $U\defeq
    U_1\cap U_2$. Sei weiter $\phi\in\contcompplus$ mit $\supp(\phi)\subset U$ und 
    sei $h\leq \sum_{i=1}^n c_i\,L_{x_i}\phi$ mit zugehörigen
    $n\in\N,\;c_i\in\R[>0],\; x_i\in G$ (wie in \cref{pf:def:covernum}) gegeben.
    
    Sei ab jetzt $j\in\{1,2\}$.
    Ist dann $x\in G$ beliebig, so können wir abschätzen:
    \[  f_j(x) = h(x) \, h_j(x) \leq \sum_{i=1}^n c_i\,(L_{x_i}\phi)(x) \, h_j(x) \]
    Falls nun für $k\in\{1,\ldots,n\}$ das betrachtete $x$ in 
    $x_k \supp(\phi)$ liegt (sodass also $(L_{x_i}\phi)(x)\neq0$ gilt), 
    so folgt aus der Wahl von $\phi$ bzw. $U$:
    \[ \abs{ h_j(x_k) - h_j(x) } < \delta,  \qtextq{also insbesondere}
        h_j(x) < h_j(x_k) + \delta
    . \]
    Greifen wir damit die Abschätzung für $f_j$ wieder auf, so folgt:
    \[ 
        f_j(x) 
        \leq \sum_{i=1}^n c_i\,(L_{x_i}\phi)(x) \, \bigl( h_j(x_i) + \delta \bigr)
    \]
    Es hängt $h_j(x_i)$ nicht mehr von $x$ ab, womit wir daraus schließen dürfen:
    \[ (f_j : \phi) \leq \sum_{i=1}^n c_i \, \bigl( h_j(x_i) + \delta \bigr) \]
    %
    Wir addieren nun die letzte Ungleichung für $j=1$ und $j=2$ und erhalten:
    \[ (f_1 : \phi) + (f_2 : \phi) 
        \leq \sum_{i=1}^n c_i \, \bigl( 
        \underbrace{ h_1(x_i) + h_2(x_i) }_{
            \hspace*{10pt} \leq 1, \text{ nach Def. v. } h_1,h_2 \hspace*{-15pt} }
        + \;\; 2\delta \bigr)
        \leq \sum_{i=1}^n c_i \, (1+2\delta)
    \]
    Da $h\leq \sum_{i=1}^n c_i\,L_{x_i}\phi$ beliebig war, folgt daraus zunächst
    \[ (f_1 : \phi) + (f_2 : \phi)  \leq  (1+2\delta) \, (h : \phi)  \]
    und nach Multiplikation mit $(f_0 : \phi)^{-1}$ auch
    \[ I_\phi(f_1) + I_\phi(f_2)  \leq  (1+2\delta) \, I_\phi(h) . \]
    Nun erhalten wir aus $h=f_1+f_2+\delta g$ mit 
    \mycref{pf:Iphiprops:scalarmult} und \ref{pf:Iphiprops:subadditive}:
    \[ 
        I_\phi(f_1) + I_\phi(f_2) 
        \leq (1+2\delta) \, \bigl(
        I_\phi(f_1+f_2) + \delta I_\phi(g) 
        \bigr)
    \]
    Verwenden wir zusätzlich \mycref{pf:Iphiprops:range}, so bekommen wir auch
    noch die unerwünschten Abhängigkeiten von $\phi$ aus der Abschätzung
    entfernt und erhalten folgende Ungleichung:
    \begin{align*}
        I_\phi(f_1) + I_\phi(f_2) 
        &\leq I_\phi(f_1+f_2) + 2\delta \, (f_1\mkern1mu{+}\mkern1mu f_2 : f_0)
        + \delta(1+2\delta) \, (g : f_0)
        \\
        &\leq I_\phi(f_1+f_2) + \epsilon
    \end{align*}
    Die letzte Relation gilt dabei auf Grund der Wahl von $\delta$ zu Beginn,
    womit die Behauptung gezeigt ist.
    \\
\end{proof}

Nach diesem nun eher technischen Beweis haben wir fast alles beisammen, um das
eigentliche Ziel dieses Abschnitts beweisen zu können. Wir verwenden im
folgenden Beweis jedoch die Charakterisierung von Kompaktheit mittels der
\emph{endlichen Durchschnittseigenschaft},\footnote{Engl.: \enquote{finite
intersection property}} sodass wir wenigstens kurz angeben, wie die verwendete
Aussage formal lautet:
%
\begin{thLemma}%
    [Charakterisierung von Kompaktheit durch Schnitte abgeschlossener Mengen]
    \label{pf:compactnessVSclosedsets}
    %
    Ist $X$ ein topologischer Raum, so ist $X$ genau dann kompakt, wenn 
    für jede Familie $(A_i)_{i\in I}$ von in $X$ abgeschlossenen Mengen gilt:
    Ist für alle endlichen $J\subset I$ der Schnitt $\bigcap_{i\in J} A_i$ nicht
    leer, so ist auch der Schnitt $\bigcap_{i\in I} A_i$ über alle Mengen der
    betrachteten Familie nicht leer.
\end{thLemma}

Der Beweis ist einfach und im Wesentlichen nur eine Umformulierung der
Definition von Kompaktheit mit Hilfe der De~Morganschen Regeln.
(Eine knappe Begründung findet sich zum Beispiel bei
Folland\cite[Ch.\,4,\;\S4.21]{bookc:folland99}.)


\begin{thSatz}[Existenz eines Haar-Maßes auf LKH-Gruppen]
    \label{pf:existence}
    %
    Ist $G$ eine LKH-Gruppe, so existiert ein Haar-Maß auf $G$.
\end{thSatz}

\begin{proof}
    Sei $G$ also eine beliebige LKH-Gruppe. Zuerst führen wir eine Abkürzung
    ein: Setze für alle $f\in\contcompplus$:
    \[ X_f \defeq \bigl[ (f_0 : f)^{-1}, \; (f : f_0) \bigr] . \]
    Wir bilden dann das (unendliche und potentiell \enquote{ziemlich große})
    Produkt
    \[ X \defeq \prod_{f\in\contcompplus} X_f =
        \prod_{f\in\contcompplus} \bigl[ (f_0 : f)^{-1}, \; (f : f_0) \bigr]
    , \]
    welches wir natürlich mit der Produkttopologie versehen. Da alle Faktoren
    kompakte Räume sind -- denn als Teilmengen von $\R$ ist nach
    \emph{Heine-Borel} die Beschränktheit und Abgeschlossenheit der $X_f$
    hinreichend zur Kompaktheit -- ist auch $X$ nach dem \emph{Satz von
    Tychonoff} ein kompakter topologischer Raum. 
    
    An dieser Stelle sollten wir kurz klären, was $X$ nun eigentlich darstellt:
    Nach Definition des Produkts, ist ein $J\in X$ eine Abbildung des Typs
    \[  J\colon\contcompplus\to\R[>0], \quad f \mapsto J(f) ,
        \qtextq{so dass}
        \forAll_{f\in\contcompplus}\colon\quad J(f) \in X_f 
    \]
    erfüllt ist. Wir erkennen also in $X$ einen Teilraum aller Funktionale auf
    $\contcompplus$.
    %
    Außerdem gilt nach \mycref{pf:Iphiprops:range}:
    \[ \forAll_{\phi\in\contcompplus}\colon\quad I_\phi \in X \]
    Definiere nun für alle Umgebungen $U\in\frakU$ von $e$:
    \[  k(U) \defeq 
        \{ I_\phi \Mid \phi\in\contcompplus,\;
                                     \supp(\phi) \subset U 
        \} \subset X
        \qqundqq
        K(U) \defeq \setclosure{k(U)} 
    . \]
    Ist $n\in\N$ und sind $U_1,\ldots,U_n\in\frakU$, so gilt offenbar:
    $K(U_1\cap \ldots \cap U_n) \subset K(U_1) \cap \ldots \cap K(U_n)$. Daraus
    folgt, dass das System aller Mengen $K(U)$ mit $U\in\frakU$ die endliche
    Durchschnittseigenschaft besitzt und somit gilt wegen der Kompaktheit von~$X$
    und \cref{pf:compactnessVSclosedsets}: 
    \[ \bigcap_{U\in\frakU} K(U)\neq\emptyset . \] Also können wir ein 
    $I\in\bigcap_{U\in\frakU} K(U)$ wählen und die Bezeichnung verrät schon,
    dass wir nun zeigen werden, dass dieses~$I$ im Wesentlichen das gesuchte
    Haar-Integral ist.
    
    Weil $I$ für alle $U\in\frakU$ im Abschluss von $k(U)$ 
    liegt, schneidet jede Umgebung von $I$ in $X$ jede dieser Mengen.
    Daraus folgt, dass wir für eine beliebige Umgebung $U\in\frakU$ von $e$ und
    endlich viele Funktionen $f_1,\ldots,f_n\in\contcompplus$ (mit $n\in\N$)
    sowie ein vorgegebenes $\epsilon\in\R[>0]$ ein $\phi\in\contcompplus$ finden
    können, so dass gilt:
    \[ \label{pf:existence:ineq} \tag{$\star$}
        \supp(\phi) \subset U \qqundqq
        \forAll_{j\in\{1,\ldots,n\}}\colon\quad 
            \abs{ I(f_j) - I_\phi(f_j) } < \epsilon
    . \]
    Wir wollen zur Klarheit den letzten Schluss noch einmal etwas genauer
    einsehen: Seien $U\in\frakU$ und $f_1,\ldots,f_n$ sowie $\epsilon$ wie oben
    gegeben. Nach Definition der Produkttopologie sind alle Projektionen
    auf die Faktoren von $X$ stetig, d.\,h. in diesem Fall sind zum Beispiel für
    alle $j\in\{1,\ldots,n\}$ die Projektionen
    \[ \pi_{f_j}\colon X \to X_{f_j} \]
    stetig. Projektion bedeutet aber hier für ein Funktional $J\in X$ nichts
    anderes, als $J$ auszuwerten, d.\,h. für $g\in\contcompplus$ gilt:
    $\pi_g(J) = J(g)$. Wählen wir nun also zu $j\in\{1,\ldots,n\}$ als Umgebung
    um $I(f_j)$ einfach  
    \[
        \bigl( I(f_j)-\epsilon,\, I(f_j)+\epsilon \bigr) \cap X_f \eqdef V_j'
    , \]
    so folgt aus der Stetigkeit von $\pi_{f_j}$, dass es eine Umgebung $V_j$ von
    $I$ in $X$ gibt, so dass $I(V_j) \subset V_j'$ erfüllt ist. Setze dann 
    $V\defeq \bigcap_{j=1}^n V_j$. Weil der Schnitt endlich ist, ist auch $V$
    eine Umgebung von $I$ in $X$ und wegen $I\in K(U)$ ist der Schnitt von $V$
    und $k(U)$ nicht leer. Nach Definition von $k(U)$ gibt es also
    ein $\phi\in\contcompplus$ mit $\supp(\phi)\subset U$, das die gewünschten
    Approximationseigenschaften erfüllt.
    
    Nachdem wir den letzten Schritt nun im Detail nachvollzogen haben, zeigen
    wir als Nächstes, dass $I$ tatsächlich linksinvariant und linear ist. Seien
    dazu $f,g\in\contcompplus$, $\lambda\in\R[>0]$ und $y\in G$. Sei außerdem
    $\epsilon\in\R[>0]$ beliebig. Dann gilt klarerweise $\lambda f
    \in\contcompplus$ und nach \cref{tg:Cctranslat} gilt auch $L_yf
    \in\contcompplus$. Weiterhin können wir nach \cref{pf:approxadditive} eine
    Umgebung $U\in\frakU$ von $e$ finden, so dass für alle
    $\phi\in\contcompplus$ mit $\supp(\phi)\subset U$ gilt:
    \[ I_\phi(f) + I_\phi(g) \leq I_\phi(f+g) + \frac{\epsilon}{4} , \]
    und da wir nach \mycref{pf:Iphiprops:subadditive} sowieso immer
    Subadditivität haben, gilt dann klarerweise:
    \[ \abs{ I_\phi(f) + I_\phi(g) - I_\phi(f+g) } \leq \frac{\epsilon}{4}
    \]
    Nach dem oben bei \eqref{pf:existence:ineq} Gezeigten können wir nun ein
    $\phi\in\contcompplus$ mit $\supp(\phi)\subset U$ wählen, so dass
    (zusätzlich) gilt:
    \[
        \abs{ I(h) - I_\phi(h) } < \frac{\epsilon}{4}
        \qtextq{für}
        h \in \{ f,\; g,\; f+g,\; \lambda f,\; L_yf \}
    . \]
    O.\,B.\,d.\,A. können wir auch noch 
    $\abs{ I(f) - I_\phi(f) } \leq \epsilon/(4\lambda)$
    annehmen (denn für $\lambda > 1$ ersetzen wir oben einfach $\epsilon/4$
    durch $\epsilon/(4\lambda) < \epsilon/4$).
    Mit Hilfe der Dreiecksungleichung und der in \cref{pf:Iphiprops} gezeigten
    Eigenschaften von $I_\phi$ erhalten wir dann die Gültigkeit der folgenden
    Ungleichungen:
    { \allowdisplaybreaks
    \begin{align*}
        \abs{ I(f) + I(g) - I(f+g) }                                        %
        &\leq \abs{ I(f) - I_\phi(f) } + \abs{ I(g) - I_\phi(g) }           \\
        &\quad+ \abs{ I_\phi(f) + I_\phi(g) - I_\phi(f+g) }                 %
            + \abs{ I_\phi(f+g) - I(f+g) }                                  \\
        &\leq 4\cdot\frac{\epsilon}{4} = \epsilon                           %
        %
        \\[2ex]
        %
        \abs{ I(\lambda f) - \lambda I(f) }                                 %
        &\leq \abs{ I(\lambda f) - I_\phi(\lambda f) }                      %
            + \abs{ \lambda I_\phi(f) - \lambda I(f) }                      \\
        &=    \abs{ I(\lambda f) - I_\phi(\lambda f) }                      %
            + \lambda \, \abs{ I_\phi(f) - I(f) }                           \\
        &\leq \frac{\epsilon}{4} + \frac{\epsilon}{4} < \epsilon            %
        %
        %\\[2ex]
        \displaybreak \\
        %
        \abs{ I(f) - I(L_yf) }                                              %
        &\leq \abs{ I(f) - I_\phi(f) } + \abs{ I_\phi(f) - I(L_yf) }        \\
        &=    \abs{ I(f) - I_\phi(f) } + \abs{ I_\phi(L_yf) - I(L_yf) }     \\
        &\leq \frac{\epsilon}{4} + \frac{\epsilon}{4} < \epsilon            %
    \end{align*} }
    Da $\epsilon\in\R[>0]$ beliebig war, erhalten wir aus den ersten beiden
    Ungleichungen die Linearität, also $I(\lambda f + g) = \lambda I(f) + I(g)$,
    und aus der letzten die Linksinvarianz, also $I(L_yf) = I(f)$.
    
    Nun setzen wir $I$ auf ganz $\contcomp(G)$ fort, indem wir zunächst einmal
    $I(0) \defeq 0$ setzen und dann wie üblich für $F\in\contcomp(G)$ definieren:
    \[ I(F) \defeq I\bigl( (\Re F)^{+} \bigr) - I\bigl( (\Re F)^{-} \bigr) 
         + i\Bigl( I\bigl( (\Im F)^{+} \bigr) - I\bigl( (\Im F)^{-} \bigr) \Bigr)
    , \]
    wobei für $F(x) = u(x) + i\,v(x)$ (mit $u(x),v(x)\in\R$) gelte:
    $\Re F \defeq u$ und $\Im F \defeq v$; sowie für
    $H\colon G\to\R$: $H^{\pm}(x) \defeq \max\bigl(0,\pm H(x)\bigr)$.
    Dabei behält $I$ offensichtlich die gerade bewiesenen Eigenschaften.
    
    Weil für alle $\phi\in\contcompplus$ aus der Definition von $I_\phi$ folgt,
    dass $I_\phi(f_0) = 1$ gilt, gilt auch $I(f_0) = 1$. (Dies ist die bei der
    Wahl von $f_0$ angesprochene Normierung. Da $f_0\in\contcompplus$
    willkürlich war, sehen wir also schon, dass es nicht \emph{das} eindeutige
    Haar-Integral geben kann.) Also ist $I\neq 0$ und außerdem nach
    Konstruktion und \mycref{pf:Iphiprops:range} positiv, d.\,h. $I(h) > 0$ für
    alle $h\in\contcompplus$. Das gewünschte Haar-Maß erhält man nun aus
    \cref{tg:rmeasuresVSfunctionals}, womit wir den Existenzbeweis abgeschlossen
    haben.
    \\
\end{proof}


\section{Eindeutigkeit des Haar-Maßes}\label{sec:haaruniq}
\vspace{-0.6\baselineskip}
\textbf{\hspace*{1.2cm}\ldots~bis auf einen konstanten Faktor.}

%Wir werden diesen Zusatz aber im Folgenden immer weglassen und einfach nur von
%der \enquote{Eindeutigkeit des Haar-Maßes} sprechen, wobei wir uns stets obigen
%Zusatz hinzudenken.

\bigskip
\noindent
Der Beweis dieser Aussage ist glücklicherweise deutlich kürzer als die
Konstruktion im letzten Abschnitt. Wir können deshalb sofort loslegen: 

\begin{thSatz}[Eindeutigkeit des Haar-Maßes (bis auf einen konstanten Faktor)]
    \label{pf:uniqueness}
    %
    Ist $G$ eine LKH-Gruppe und sind $\mu$ und $\nu$ zwei Haar-Maße auf $G$,
    dann existiert ein $c\in\R[>0]$, so dass $\mu = c\,\nu$ gilt.
\end{thSatz}

\begin{proof}
    Seien $G,\mu,\nu$ wie in der Behauptung. Wir zeigen zunächst, dass es eine
    Konstante $c\in\R[>0]$ gibt, so dass
    \[ c = \frac{\int_G f \dif\mu}{\int_G f \dif\nu} \]
    für alle $f\in\contcompplus$ gilt. Daraus wird dann direkt die Behauptung
    folgen. Seien also $f,g\in\contcompplus$ und $\epsilon\in\R[>0]$
    beliebig. Wir wählen dann eine feste kompakte und symmetrische Umgebung
    $U\in\frakU$ von $e$. (Dies ist sicher möglich, denn man kann eine
    kompakte Umgebung wählen, da $G$ lokalkompakt ist, und diese dann z.\,B. mit
    ihrem Bild unter Invertierung schneiden.) 
    Definiere nun Abbildungen für $y\in U$ wie folgt:
    \[  \tilde f_y \defeq R_yf - L_{y^{-1}}f
        \qqundqq
        \tilde g_y \defeq R_yg - L_{y^{-1}}g
    . \]
    Aus \cref{tg:Cctranslat} folgt, dass auch $\tilde f_y,\tilde g_y$ Funktionen
    aus $\contcomp(G)$ sind. Nach \cref{tg:unicont} gibt es Umgebungen
    $U_R,U_L\in\frakU$ von $e$, so dass für alle $y\in U_R$ bzw. $y\in U_L$
    gilt: 
    $\supnorm{f-R_yf}<\epsilon/2$ bzw. $\supnorm{f-L_{y^{-1}}f}<\epsilon/2$.
    Analoges gilt mit $g$ statt $f$ für $U_R',U_L'\in\frakU$.
    Setzte dann $U_R\cap U_L\cap U_L'\cap U_R' \eqdef U \in\frakU$
    und wähle eine (nach \mycref{tg:basics:symV} existierende) symmetrische
    Umgebung $V\in\frakU$ mit $V\subset U$. Dann gilt für $y\in V$ nach
    Konstruktion und der Dreiecksungleichung:
    \begin{align*}
        \supnorm{ \tilde f_y }                                          %
        &=    \supnorm{ R_yf - L_{y^{-1}}f }                            \\
        &\leq \supnorm{ R_yf - f } + \supnorm{ f - L_{y^{-1}}f }        \\
        &\leq \frac{\epsilon}{2} + \frac{\epsilon}{2} = \epsilon        %
        \\[1ex]
        \supnorm{ \tilde g_y }                                          %
        &=    \supnorm{ R_yg - L_{y^{-1}}g }                            %
         \leq \epsilon                                                  %
    \end{align*}
    Wir wählen nun ein $h\in\contcompplus$ mit $\supp(\phi)\subset V$, wobei $h$
    zusätzlich symmetrisch in folgendem Sinne sein soll: $\forall\,x\in G\colon
    h(x) = h(x^{-1})$. (Man kann so ein $h$ immer finden, denn hat man ein~$\tilde h$,
    welches die letzte Bedingung noch nicht erfüllt, so setze einfach 
    $h(x) \defeq \tilde h(x) + \tilde h(x^{-1})$, was wegen der Symmetrie von
    $V$ immer noch einen Träger innerhalb von $V$ besitzt.)
    
    Wegen der Linksinvarianz von $\mu$ gilt dann:\footnote{Da wir jedes Mal über ganz
    $G$ integrieren, lassen wir dies in der Notation nun weg.}
    \begin{align*}
        \left( \int h \dif\nu \right) \left( \int f \dif\mu \right)     %
        &= \iint h(y) \, f(x) \dif{\mu(x)} \dif{\nu(y)}                 \\
        &= \iint h(y) \, f(yx) \dif{\mu(x)} \dif{\nu(y)}                %%
        %
        \label{pf:uniq:firstint} \tag{$\triangle_1$}
    \end{align*}
    Man rechnet leicht nach, dass für $f_1,f_2\in\contcomp(G)$ und
    $\Phi(x,y)\defeq f_1(x)\,f_2(y)$ für die Träger gilt: 
    \[ \supp(\Phi) \subset \supp(f_1) \times \supp(f_2)
    . \]
    In Kombination mit \cref{tg:Cctranslat} ergibt sich, dass wir in den
    folgenden Formeln zu Recht die Integrale vertauschen dürfen, da alle
    Funktionen kompakten Träger haben und somit \cref{tg:Ccfubini} greift.
    Damit gilt nun also mit der Symmetrie von~$h$ und der Linksinvarianz von
    $\mu$ und $\nu$:
    \begin{align*} 
        \left( \int h \dif\mu \right) \left( \int f \dif\nu \right)     %
        &= \iint h(x) \, f(y) \dif{\mu(x)} \dif{\nu(y)}                 \\
        &= \iint h(y^{-1}x) \, f(y) \dif{\mu(x)} \dif{\nu(y)}           \\
        &= \iint h(x^{-1}y) \, f(y) \dif{\mu(x)} \dif{\nu(y)}           \\
        &= \iint h(x^{-1}y) \, f(y) \dif{\nu(y)} \dif{\mu(x)}           \\
        &= \iint h(y) \, f(xy) \dif{\nu(y)} \dif{\mu(x)}                \\
        &= \iint h(y) \, f(xy) \dif{\mu(x)} \dif{\nu(y)}                %%
        %
        \label{pf:uniq:secondint} \tag{$\triangle_2$}
    \end{align*}
    Nehmen wir nun die beiden Gleichheiten \eqref{pf:uniq:firstint} und 
    \eqref{pf:uniq:secondint} zusammen und subtrahieren die zweite von der
    ersten, so ergibt sich:
    \begin{align*}
        \abs*{                                                              %
                \int h \dif\nu \; \int f \dif\mu                            %
            -   \int h \dif\mu \; \int f \dif\nu                            %
        }                                                                   %
        &= \abs*{ \iint h(y) \,                                             %
            \bigl( f(xy) - f(yx) \bigr) \dif{\mu(x)} \dif{\nu(y)} }         \\
        &\leq \iint h(y) \,                                                 %
                \abs[\big]{\tilde f_y(x)} \dif{\mu(x)} \dif{\nu(y)}         \\
        &\leq \iint h(y) \;\epsilon\mkern2mu\chi_{\supp(\tilde f_y)}(x)     %
                \dif{\mu(x)} \dif{\nu(y)}                                   \\
        &\leq \epsilon \,                                                   %
            \underbrace{ \mu\Bigl( \supp(f)\,U \;\cup\; U\supp(f)           %
            \Bigr) }_{ \eqdef \alpha }                                      %
            \, \int h \dif\nu                                               %%
    \end{align*}
    In der vorletzten Zeile haben wir ausgenutzt, dass der Träger von $h$ in $V$
    enthalten ist, und die letzte Abschätzung erhalten wir folgendermaßen:
    klarerweise gilt $\supp(\tilde f_y) \subset \supp(f)\,U \,\cup U\supp(f)$, wobei die
    rechte Menge nach \mycref{tg:basics:KKcompact} kompakt ist; also ist
    auch ihr Maß endlich und positiv, denn $\mu$ ist lokal-endlich. Außerdem ist
    auch das Integral über $h$ be\-züg\-lich~$\nu$ nach \mycref{tg:haarbasics:int}
    von null verschieden (und endlich).
    
    Nun führen wir exakt dieselben Rechnungen für $g$ statt $f$ durch, woraus
    sich die analoge Abschätzung für $g$ ergibt:
    \begin{equation*}
        \abs*{
            \int h \dif\nu \; \int g \dif\mu
            -   \int h \dif\mu \; \int g \dif\nu
        }
        \leq \epsilon \,
            \underbrace{ \mu\Bigl( \supp(g)\,U \;\cup\; U\supp(g)           %
            \Bigr) }_{ \eqdef \beta }                                       %
            \, \int h \dif\nu                                               %%
    \end{equation*}
    Wir dividieren die Ungleichung für $f$ durch 
    $\int h \dif[\,]\nu \, \int f \dif[\,]\nu > 0$ und analog die für $g$ durch 
    $\int h \dif[\,]\nu \, \int g \dif[\,]\nu > 0$, womit wir folgende Abschätzungen
    bekommen:
    \[  \abs*{       \frac{\int f \dif\mu}{\int f \dif\nu} 
                 -   \frac{\int h \dif\mu}{\int h \dif\nu}
             }
             \leq \frac{\epsilon \, \alpha}{\int f \dif\nu}
        \qqundqq
        \abs*{       \frac{\int g \dif\mu}{\int g \dif\nu} 
                 -   \frac{\int h \dif\mu}{\int h \dif\nu}
             }
             \leq \frac{\epsilon \, \beta}{\int g \dif\nu}
    . \]
    Aus beiden zusammen und der Dreiecksungleichung folgt:
    \[  \abs*{       \frac{\int f \dif\mu}{\int f \dif\nu} 
                 -   \frac{\int g \dif\mu}{\int g \dif\nu}
             }
             \leq \frac{\epsilon \, \alpha}{\int f \dif\nu}
                + \frac{\epsilon \, \beta}{\int g \dif\nu}
    \]
    Da aber $\epsilon$ beliebig war, folgt dann schon:
    \[ \frac{\int f \dif\mu}{\int f \dif\nu} 
        = \frac{\int g \dif\mu}{\int g \dif\nu}
    \]
    Wählen wir nun ein festes $g_0\in\contcompplus$, so gilt also für
    $c \defeq \int g_0 \dif\mu\, / \int g_0 \dif\nu$ und alle $f\in\contcompplus$:
    \[ \int f \dif\mu = c  \int f \dif\nu \]
    Dies impliziert aber natürlich wegen der Linearität des Integrals, dass
    diese Gleichheit auch schon für alle $f\in\contcomp(G)$ gelten muss.
    Aus der Eindeutigkeitsaussage in \cref{tg:rmeasuresVSfunctionals} bzw. der
    im Riesz'schen Derstellungssatz \pref{tg:riesz} folgt dann:
    $\mu = c\,\nu$, und das wollten wir zeigen.
    \\
\end{proof}



















\chapter{Existenz und Eindeutigkeit des Haar'schen Maßes}
Wir wollen nun die Existenz und Eindeutigkeit (bis auf einen positiven Faktor)
eines Haar-Maßes auf LKH-Gruppen zeigen. 
% (Im Folgenden meinen wir mit \enquote{Eindeutigkeit des Haar-Maßes} 
% immer die Eindeutigkeit bis auf einen konstanten Faktor.)
Für den Existenzbeweis halten wir uns an den erstmals 1940 von 
\emph{Andr\'e Weil} in allgemeiner Form veröffentlichten Beweis für beliebige
LKH-Gruppen, wie man ihn etwa bei
Elstrod\cite[Kap.\,VIII,\;\S3,\;3.12]{bookc:elstrod11} 
oder auch bei Folland\cite[\S11,\;11.8]{bookc:folland99} findet.
Dieser benutzt das \emph{Auswahlaxiom} in Form des
\emph{Satzes von Tychonoff}, weshalb es an dieser Stelle erwähnenswert ist, dass
\emph{Henri Cartan} (ebenfalls 1940) auch einen Beweis gefunden hat, der ohne
selbiges auskommt. (Siehe dazu auch: Alfsen\cite{artcle:alfsen63}.)
Für den Eindeutigkeitsbeweis verwenden wir ein Argument, das hauptsächlich auf
dem Gebrauch des \emph{Satzes von Fubini} beruht, siehe beispielsweise
Folland\cite[\S11,\;11.9]{bookc:folland99}. Auch hier gibt es alternative
Beweise, die mit elementareren Mitteln auskommen. Der ursprüngliche Beweis von
\emph{A. Weil} macht zum Beispiel keinen Gebrauch vom Satz von Fubini und ein
Beweis nach diesem Schema findet man im oben schon zitierten Satz bei Elstrod.

\medskip
Da wir schon gezeigt haben, dass sich Haar-Maße und Haar-Integrale entsprechen,
genügt es, die Existenz und (im Wesentlichen) Eindeutigkeit eines Haar-Integrals
zu zeigen, und dies ist auch der übliche Ansatz. Im Übrigen meinen wir hier und
im Rest dieses Kapitels stets \emph{linkes} Haar-Maß und \emph{linkes}
Haar-Integral, wenn wir es nicht explizit spezifizieren. Um nun eine Motivation
für die weiter unten auftrende Konstruktion zu geben, betrachten wir folgendes
Szenario: Sei $G$ eine LKH-Gruppe und seien $K\subset G$ eine kompakte und
$U\in\frakU$ eine offene Menge. Weil wir ein von innen reguläres Maß suchen,
genügt es prinzipiell, das Maß auf kompakten Mengen zu kennen. Wir möchten nun
also ein \enquote{Maß} für $K$ angeben, was wir zum Beispiel unter Verwendung
von $U$ wie folgt tun können:
Weil $K$ kompakt ist, finden wir ein $n\in\N$ und endlich viele $x_i\in G$, so
dass $(x_iU)_{i\in\{1,\ldots,n\}}$ eine Überdeckung von~$K$ bildet. Definiert
man nun $(K : U)$ als das minimale derartige $n\in\N$, so misst $(K : U)$ in
gewissem Rahmen wie groß $K$ in Relation zu $U$ ist. Die Idee ist nun, $U$ immer
kleiner zu wählen, so dass $U$ auf $\{e\}$ zusammenschrumpft und man somit die
Menge $K$ mit immer weniger \enquote{Überlappungen} überdecken kann. Mittels
einer Normierung, also $(K : U)/(K_0 : U)$ für festes $K_0\subset G$, können wir
erhoffen, dass sich bei dem eben skizzierten Grenzprozess ein Wert für diesen
Quotienten einstellt und wir dadurch ein (offenbar linksinvariantes) Maß auf~$G$
bekommen.




















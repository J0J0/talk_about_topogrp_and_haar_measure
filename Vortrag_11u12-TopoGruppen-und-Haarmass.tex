\documentclass[11pt,a4paper,ngerman,DIV=11]{scrreprt}

%%%%%%%%%%%%%%%%%%%%%%%%%%%%%%%%%%%%%%%%%%%%%%%%%%%%%%%%%%%%%%%%%%%%%
%%% packages
%%%%%%%%%%%%%%%%%%%%%%%%%%%%%%%%%%%%%%%%%%%%%%%%%%%%%%%%%%%%%%%%%%%%%

\usepackage[utf8]{inputenc}
\usepackage[T1]{fontenc}
\usepackage[ngerman]{babel}

\usepackage{amsmath}
\usepackage{amssymb}
\usepackage{amsthm}
\usepackage{mathtools}

\usepackage[babel]{csquotes}
\usepackage[shortlabels]{enumitem}
\usepackage[numbers,sort&compress]{natbib}
\usepackage{ifmtarg}

%\usepackage[pdftex]{graphicx}
%\usepackage[all]{xy}

\usepackage[pdftex,bookmarks,colorlinks=false,pdfborder={0 0 0},%
            pdftitle={Seminar Modultheorie - Vortrag 11 und 12: Topologische
            Gruppen und das Haar'sche Maß},%
            pdfauthor={Johannes Prem}]{hyperref}
%
\usepackage{cleveref}

\usepackage{helpers} % my own helpers.sty


%%%%%%%%%%%%%%%%%%%%%%%%%%%%%%%%%%%%%%%%%%%%%%%%%%%%%%%%%%%%%%%%%%%%%
%%% macro definitions and other things
%%%%%%%%%%%%%%%%%%%%%%%%%%%%%%%%%%%%%%%%%%%%%%%%%%%%%%%%%%%%%%%%%%%%%

% global redefinition of paragraph spacing
\setlength{\parindent}{0pt}
\setlength{\parskip}{0.5em}

% make parenthesized versions of \ref and cleveref's \cref
\newcommand*{\pref}[1]{(\ref{#1})}
\newcommand*{\pcref}[1]{(\cref{#1})}

% make \varepsilon and \varphi default
\varifygreekletters{\epsilon\phi}

% change the qedsymbol to my favoured blacksquare
\renewcommand{\qedsymbol}{$\blacksquare$}

% style for /all/ theorem like environments
\newtheoremstyle{mythms}
 {15pt}% space above
 {12pt}% space below 
 {}% body font
 {}% indent amount
 {\bfseries}% theorem head font
 {.}% punctuation after theorem head
 {0.6cm}% space after theorem head (\newline possible)
 {}% theorem head spec 
 
% set style and define thm like environments
\theoremstyle{mythms}
\newtheorem{globalnum}{DUMMY DUMMY DUMMY}[chapter]
\newtheorem{thDef}[globalnum]{Definition}
\newtheorem{thSatz}[globalnum]{Satz}
%\newtheorem{thPropos}[globalnum]{Proposition}
\newtheorem{thLemma}[globalnum]{Lemma}
\newtheorem{thKorollar}[globalnum]{Korollar}

\newtheorem{thBemerkung}[globalnum]{Bemerkung}
\newtheorem{thBeisp}[globalnum]{Beispiel}
\newtheorem{thBeispiele}[globalnum]{Beispiele}
\newenvironment{BspList}{%
\nopagebreak\begin{thBeispiele}%
\hfill\begin{enumerate}[a),parsep=0pt,itemsep=0.8ex,leftmargin=2em]%
}{%
\end{enumerate}\end{thBeispiele}
}
%

% also define a 'proofsketch' version of 'proof'
\newenvironment{proofsketch}[1][]{%
\begin{proof}[Beweisskizze#1]
}{%
\end{proof}
}

% inject pdfbookmarks at thm like environments
\makeatletter
\let\origthmhead=\thmhead
\renewcommand{\thmhead}[3]{%
\origthmhead{#1}{#2}{#3}%
\pdfbookmark[1]{#1\@ifnotempty{#1}{ }#2\thmnote{ (#3)}}{#1#2}%
}
\makeatother

% define some additional 'operators'
\DeclareMathOperator*{\Exists}{\exists}
\DeclareMathOperator*{\forAll}{\forall}
%\DeclareMathOperator{\Kern}{ker}
%\DeclareMathOperator{\Image}{im}

% define an 'abs' and 'Spann' command
\DeclarePairedDelimiter{\abs}{\lvert}{\rvert}
\DeclarePairedDelimiter{\Spann}{\langle}{\rangle}

% define missing arrows
\newcommand{\longto}{\longrightarrow}
\newcommand{\longhookrightarrow}{\lhook\joinrel\relbar\joinrel\rightarrow}

% provide mathbb symbols \N \Z \Q \R and \C
\defmathbbsymbols{N Z Q R C}

% just some shortcuts
\newcommand{\mr}{\mathrm}
%\newcommand{\barfrak}[1]{\bar{\mathfrak{#1}}}
%\newcommand{\len}[1][R]{\modulelength_{#1}}
\newcommand{\ZRest}[1]{\Z/#1\Z}
%\newcommand{\D}{\mr{d}}
%\newcommand{\A}[2]{a_{#1#2}}
%\newcommand{\vzfix}{\phantom{-}}


% listing with -- is nicer than with bullets 
\setlist[itemize,1]{label=--}

%% xy tip selection (ComputerModern)
%%\SelectTips{cm}{}
%%\UseTips

% start at chapter 0
\setcounter{chapter}{-1}

%%%%%%%%%%%%%%%%%%%%%%%%%%%%%%%%%%%%%%%%%%%%%%%%%%%%%%%%%%%%%%%%%%%%%
%%% document
%%%%%%%%%%%%%%%%%%%%%%%%%%%%%%%%%%%%%%%%%%%%%%%%%%%%%%%%%%%%%%%%%%%%%

\begin{document}


\subject{Seminar: Maßtheorie}
\title{Topologische Gruppen\\ und Haar'sches Maß}
\author{Johannes Prem}
\date{05.04.2013}

\maketitle


\chapter{Vorwort, Notation und Konventionen}
Dieses Skript behandelt \emph{Haar'sche Maße} auf \emph{topologischen Gruppen}.
Dazu wird zunächst der letztere Begriff eingeführt und anhand einiger Beispiele
verdeutlicht. Dann werden einige einfache Resultate gezeigt und schließlich der
Begriff des \emph{Haar-Maßes} eingeführt. Zuletzt wird die Existenz und
(im Wesentlichen) Eindeutigkeit des Haar-Maßes auf \emph{lokalkompakten
hausdorffschen topologischen Gruppen} gezeigt und als erste Anwendung die
\emph{modulare Funktion} besprochen.


\bigskip
In diesem Skript wird folgende Notation verwendet:
\begin{itemize}
    \item
        In Analogie zu der in englischer Literatur manchmal zu findenden
        Abkürzung \enquote{LCH space} werden wir für eine topologische Gruppe,
        die außerdem lokalkompakt und hausdorffsch ist, die Abkürzung
        \emph{LKH-Gruppe} verwenden.
        
    \item
        Sowohl $\subset$ als auch $\subseteq$ stehen für: enthalten oder gleich.
        Echt enthalten wird durch $\subsetneq$ gekennzeichnet.
    
    \item
        Die \emph{Natürlichen Zahlen $\N$} beginnen mit $0$.
    
    %\item
    %    Zu einem Ring $R$ bezeichnet $R^\times$ die Einheitengruppe des Rings.
\end{itemize}


\bigskip
Weiterhin vereinbaren wir für einige Begriffe, die in der Literatur
unterschiedlich definiert werden, Folgendes% (wobei $(X,\Topo)$ ein beliebiger topologischer Raum sei)
:
\begin{itemize}
    \item 
        Ein topologischer Raum ist \emph{lokalkompakt} genau dann,
        wenn jeder Punkt eine kompakte Umgebung besitzt.
        
    \item
        Ein topologischer Raum ist \emph{regulär} genau dann,
        wenn er die Trennungsaxiome $T_2$ und $T_3$ erfüllt, d.\,h. wenn er ein
        Hausdorff-Raum ist und je eine (nicht-leere) abgeschlossene Menge und 
        ein Punkt aus dem Komplement selbiger durch (offene) Umgebungen
        getrennt werden können. (Man zeigt leicht, dass es schon genügt, 
        statt $T_2$ nur $T_0$ zu fordern, um eine äquivalente Definition zu
        erhalten.)
\end{itemize}


\nocite{bookc:folland99}
\nocite{bookc:elstrod11}

\appendix
\bibliographystyle{plaindin}
\bibliography{bibsources}

\end{document}


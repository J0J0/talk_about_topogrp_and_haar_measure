\documentclass[11pt,a4paper,ngerman,DIV=11,bibliography=totoc]{scrreprt}

%%%%%%%%%%%%%%%%%%%%%%%%%%%%%%%%%%%%%%%%%%%%%%%%%%%%%%%%%%%%%%%%%%%%%
%%% packages
%%%%%%%%%%%%%%%%%%%%%%%%%%%%%%%%%%%%%%%%%%%%%%%%%%%%%%%%%%%%%%%%%%%%%

\usepackage[utf8]{inputenc}
\usepackage[T1]{fontenc}
\usepackage[ngerman]{babel}

\usepackage{amsmath}
\usepackage{amssymb}
\usepackage{amsthm}
\usepackage{mathtools}

\usepackage[babel]{csquotes}
\usepackage[shortlabels]{enumitem}
\usepackage[numbers,sort&compress]{natbib}
\usepackage{ifmtarg}
\usepackage{xstring}
\usepackage{remreset}

%\usepackage[pdftex]{graphicx}
%\usepackage[all]{xy}

\usepackage[pdftex,bookmarks,colorlinks=false,pdfborder={0 0 0},%
            pdfauthor={Johannes Prem}]{hyperref}
%
\usepackage{cleveref}

\usepackage{helpers} % my own helpers.sty

% 'ß' in the title requires this to be called outside \usepackage's options
\hypersetup{%
    pdftitle={Seminar Maßtheorie - Vortrag 11 und 12: Topologische
              Gruppen und das Haar'sche Maß}}

%%%%%%%%%%%%%%%%%%%%%%%%%%%%%%%%%%%%%%%%%%%%%%%%%%%%%%%%%%%%%%%%%%%%%
%%% macro definitions and other things
%%%%%%%%%%%%%%%%%%%%%%%%%%%%%%%%%%%%%%%%%%%%%%%%%%%%%%%%%%%%%%%%%%%%%

% global redefinition of paragraph spacing
%\setlength{\parindent}{0pt}
%\setlength{\parskip}{0.5em}

% don't reset footnote numbers
\makeatletter
\@removefromreset{footnote}{chapter}
\makeatother


% make parenthesized versions of \ref and cleveref's \cref
\newcommand*{\pref}[1]{(\ref{#1})}
\newcommand*{\pcref}[1]{(\cref{#1})}

% make a even more clever \mycref that produces "Lemma 42a)" etc.
\newcommand{\mycref}[1]{%
    \begingroup%
    \StrCount{#1}{:}[\mycrefCount]%
    \StrBefore[\mycrefCount]{#1}{:}[\myrefMain]%
    \expandafter\cref\expandafter{\myrefMain}\,\ref{#1}%
    \endgroup%
}

% make \varepsilon and \varphi default
\varifygreekletters{\epsilon\phi}

% change the qedsymbol to my favoured blacksquare
\renewcommand{\qedsymbol}{$\blacksquare$}

% style for /all/ theorem like environments
\newtheoremstyle{mythms}
 {15pt}% space above
 {12pt}% space below 
 {}% body font
 {}% indent amount
 {\bfseries}% theorem head font
 {.}% punctuation after theorem head
 {0.6cm plus 0.25cm minus 0.1cm}% space after theorem head (\newline possible)
 {}% theorem head spec 
 
% set style and define thm like environments
\theoremstyle{mythms}
\newtheorem{globalnum}{DUMMY DUMMY DUMMY}[chapter]
\newtheorem{thDef}[globalnum]{Definition}
\newtheorem{thNotation}[globalnum]{Notation}
\newtheorem{thSatz}[globalnum]{Satz}
%\newtheorem{thPropos}[globalnum]{Proposition}
\newtheorem{thLemma}[globalnum]{Lemma}
\newtheorem{thKorollar}[globalnum]{Korollar}

\newtheorem{thBemerkung}[globalnum]{Bemerkung}
\newtheorem{thBeisp}[globalnum]{Beispiel}
\newtheorem{thBeispiele}[globalnum]{Beispiele}
\newenvironment{BspList}[1][]{%
\nopagebreak\begin{thBeispiele}#1%
\hfill\begin{enumerate}[a),parsep=0pt,itemsep=0.8ex,leftmargin=2em]%
}{%
\end{enumerate}\end{thBeispiele}
}
%

% also define a 'proofsketch' version of 'proof'
\newenvironment{proofsketch}[1][]{%
\begin{proof}[Beweisskizze#1]
}{%
\end{proof}
}

% inject pdfbookmarks at thm like environments
\makeatletter
\let\origthmhead=\thmhead
\renewcommand{\thmhead}[3]{%
\origthmhead{#1}{#2}{#3}%
\pdfbookmark[2]{#1\@ifnotempty{#1}{ }#2\thmnote{ (#3)}}{#1#2}%
}
\makeatother

% new math operators
\DeclareMathOperator*{\bigdotcup}{\overset{\mkern0mu\scalebox{0.6}{$\bullet$}}{\bigcup}}

% new math 'operators'
\DeclareMathOperator{\matGL}{GL}
\DeclareMathOperator{\matSL}{SL}
\DeclareMathOperator{\matO}{O}
\DeclareMathOperator{\matSO}{SO}
\DeclareMathOperator{\supp}{supp}
\DeclareMathOperator{\id}{id}
\DeclareMathOperator{\J}{J\mkern-1mu}
\DeclareMathOperator{\Potmenge}{\mathcal{P}}
%\DeclareMathOperator{\Kern}{ker}
%\DeclareMathOperator{\Image}{im}

% make quantors that use \limits per default
\DeclareMathOperator*{\Exists}{\exists}
\DeclareMathOperator*{\forAll}{\forall}

% define an 'abs', 'norm' and 'Spann' command
\DeclarePairedDelimiter{\abs}{\lvert}{\rvert}
\DeclarePairedDelimiter{\norm}{\lVert}{\rVert}
\DeclarePairedDelimiter{\Spann}{\langle}{\rangle}

%---
% the following creates an operator norm as a tripple stroke \vert
% (source: mathabx fonts)
\DeclareFontFamily{U}{matha}{\hyphenchar\font45}
\DeclareFontShape{U}{matha}{m}{n}{
      <5> <6> <7> <8> <9> <10> gen * matha
      <10.95> matha10 <12> <14.4> <17.28> <20.74> <24.88> matha12
      }{}
\DeclareSymbolFont{matha}{U}{matha}{m}{n}
\DeclareFontFamily{U}{mathx}{\hyphenchar\font45}
\DeclareFontShape{U}{mathx}{m}{n}{
      <5> <6> <7> <8> <9> <10>
      <10.95> <12> <14.4> <17.28> <20.74> <24.88>
      mathx10
      }{}
\DeclareSymbolFont{mathx}{U}{mathx}{m}{n}

\DeclareMathDelimiter{\vvvert}{0}{matha}{"7E}{mathx}{"17}
\DeclarePairedDelimiter{\opnorm}{\vvvert}{\vvvert}
%---

% define missing arrows
\newcommand{\longto}{\longrightarrow}
\newcommand{\longhookrightarrow}{\lhook\joinrel\relbar\joinrel\rightarrow}

% provide mathbb symbols \N \Z \Q \R and \C
\defmathbbsymbols{N Z Q C}
\defmathbbsymbolsubs{R}

% define some set specific macros
\newcommand{\setclosure}[1]{\overline{#1}}
\newcommand{\setinterior}[1]{#1^\circ}
\newcommand{\setboundary}[1]{\partial #1}

% just some shortcuts
\newcommand{\mr}{\mathrm}
\newcommand{\mt}{^\mathsf{t}}
\newcommand{\defeq}{\coloneqq}
\newcommand{\eqdef}{\eqqcolon}
%\newcommand{\barfrak}[1]{\bar{\mathfrak{#1}}}
%\newcommand{\len}[1][R]{\modulelength_{#1}}
%\newcommand{\ZRest}[1]{\Z/#1\Z}
%\newcommand{\D}{\mr{d}}
%\newcommand{\A}[2]{a_{#1#2}}
%\newcommand{\vzfix}{\phantom{-}}
\newcommand{\qiffq}{\quad\iff\quad}
\newcommand{\qimpliesq}{\quad\implies\quad}
\newcommand{\qoderq}{\qtextq{oder}}
\newcommand{\qqundqq}{\qqtextqq{und}}
\newcommand{\qundq}{\qtextq{und}}
\newcommand{\qqtextqq}[1]{\qquad\text{#1}\qquad}
\newcommand{\qtextq}[1]{\quad\text{#1}\quad}
\newcommand{\supnorm}[1]{\norm{#1}_\infty}
\newcommand{\borelsigmaalg}{\mathcal{B}}
\newcommand{\dif}[2][\;]{#1\mathrm{d} #2}
\newcommand{\half}{\frac{1}{2}}
\newcommand{\thalf}{\tfrac{1}{2}}
\newcommand{\Nfolge}[1]{\left(#1_n\right)_{n\in\N}}
\newcommand{\pot}[1]{\Potmenge(#1)}


\newcommand{\frakU}{\mathfrak{U}}

\newcommand{\continuous}{C}
\newcommand{\contcomp}{C_c}
\newcommand{\contcompplus}{C_c^+}

\DeclareMathOperator{\Realteil}{Re}
\DeclareMathOperator{\Imaginaerteil}{Im}
\let\Re=\Realteil
\let\Im=\Imaginaerteil


% make a \Mid macro as flexible replacement for \mid in sets definitions
\newcommand{\Mid}[1][\,]{%
#1%
\ifnum\currentgrouptype=16%
\middle\vert\else\vert\fi%
#1%
}
% \dots and a version for custom size control
\newcommand{\cMid}[2][\,]{%
#1#2\vert#1%
}

% listing with -- is nicer than with bullets 
\setlist[itemize,1]{label=--}

%% xy tip selection (ComputerModern)
%%\SelectTips{cm}{}
%%\UseTips

% start at chapter 0
\setcounter{chapter}{-1}

%%%%%%%%%%%%%%%%%%%%%%%%%%%%%%%%%%%%%%%%%%%%%%%%%%%%%%%%%%%%%%%%%%%%%
%%% document
%%%%%%%%%%%%%%%%%%%%%%%%%%%%%%%%%%%%%%%%%%%%%%%%%%%%%%%%%%%%%%%%%%%%%

\begin{document}


\subject{Seminar: Maßtheorie}
\title{Topologische Gruppen\\ und Haar'sches Maß}
\author{Johannes Prem}
\date{05.04.2013}

\maketitle


\chapter{Vorwort, Notation und Konventionen}
Dieses Skript behandelt \emph{Haar'sche Maße} auf \emph{topologischen Gruppen}.
Dazu wird zunächst der letztere Begriff eingeführt und anhand einiger Beispiele
verdeutlicht. Dann werden einige einfache Resultate gezeigt und schließlich der
Begriff des \emph{Haar-Maßes} eingeführt. Zuletzt wird die Existenz und
(im Wesentlichen) Eindeutigkeit des Haar-Maßes auf \emph{lokalkompakten
hausdorffschen topologischen Gruppen} gezeigt und als erste Anwendung die
\emph{modulare Funktion} besprochen.


\bigskip
In diesem Skript wird folgende Notation verwendet:
\begin{itemize}
    \item
        In Analogie zu der in englischer Literatur manchmal zu findenden
        Abkürzung \enquote{LCH space} werden wir für eine topologische Gruppe,
        die außerdem lokalkompakt und hausdorffsch ist, die Abkürzung
        \emph{LKH-Gruppe} verwenden.
        
    \item
        Sowohl $\subset$ als auch $\subseteq$ stehen für: enthalten oder gleich.
        Echt enthalten wird durch $\subsetneq$ gekennzeichnet.
    
    \item
        Die \emph{Natürlichen Zahlen $\N$} beginnen mit $0$.
    
    \item
        Zu einem Ring $R$ bezeichnet $R^\times$ die Einheitengruppe des Rings.
\end{itemize}


\bigskip
Weiterhin vereinbaren wir für einige Begriffe, die in der Literatur
unterschiedlich definiert werden, Folgendes% (wobei $(X,\Topo)$ ein beliebiger topologischer Raum sei)
:
\begin{itemize}
    \item
        Eine \emph{Umgebung} eines Punkts braucht nicht offen zu sein, sie muss
        nur eine eine \emph{offene Umgebung} des Punkts enthalten.
        
    \item 
        Ein topologischer Raum ist \emph{lokalkompakt} genau dann,
        wenn jeder Punkt eine kompakte Umgebung besitzt.
        
    \item
        Ein topologischer Raum ist \emph{regulär} genau dann,
        wenn er die Trennungsaxiome $T_2$ und $T_3$ erfüllt, d.\,h. wenn er ein
        Hausdorff-Raum ist und je eine (nicht-leere) abgeschlossene Menge und 
        ein Punkt aus dem Komplement selbiger durch (offene) Umgebungen
        getrennt werden können. (Man zeigt leicht, dass es schon genügt, 
        statt $T_2$ nur $T_0$ zu fordern, um eine äquivalente Definition zu
        erhalten.)
\end{itemize}

\chapter{Topologische Gruppen}
Für eine Gruppe $G$ bezeichne stets $e$ das neutrale Element. Weiter werden wir
im Folgenden die auftretenden Gruppen meist multiplikativ schreiben und dabei
wie üblich das explizite Multiplikationszeichen unterdrücken. Das Inverse zu
einem Element $a\in G$ bezeichnen wir mit $a^{-1}$.

\begin{thDef}[Topologische Gruppe]
    Eine Gruppe $(G,\cdot, e)$ ist eine \emph{topologische Gruppe}, wenn auf ihr
    zusätzlich eine Topologie erklärt ist, für welche die Gruppenoperationen
    \begin{alignat*}{2}
        G \times G &\to G,  &\quad (a,b) &\mapsto ab
        \\
        G &\to G,           &\quad     a &\mapsto a^{-1}
    \end{alignat*}
    stetig sind. Dabei wird $G\times G$ mit der Produkttopologie versehen.
\end{thDef}

Sprechen wir im Folgenden davon, dass eine topologische Gruppe irgendeine
topologische Eigenschaft besitzt, so beziehen wir diese immer auf den
unterliegenden topologischen Raum.
Ist $G$ eine Gruppe und sind $A,B\subset G$ Teilmengen von $G$ sowie $x\in G$
ein Element von $G$, so benutzen wir außerdem folgende (intuitive) Notationen:
\begin{thNotation} 
    \begin{align*}
        xA &\defeq \{ xa \Mid a\in A \},            &
        Ax &\defeq \{ ax \Mid a\in A \},            \\
        AB &\defeq \{ ab \Mid a\in A,\, b\in B \},  &
        A^{-1} &\defeq \{ a^{-1} \Mid a\in A \}     %
    \end{align*}
\end{thNotation}

Gilt für eine Teilmenge $A\subset G$ die Gleichheit $A=A^{-1}$, so nennen wir
$A$~\emph{symmetrisch}. Nun geben wir aber zunächst einmal ein paar Beispiel an:
\begin{BspList}
\item
    Eine beliebige Gruppe $G$ ist bezüglich der diskreten Topologie eine
    topologische Gruppe. Diese ist dann trivialerweise eine LKH-Gruppe.
    
\item
    Es sind (für $n\in\N$) $(\R^n,+),\; (\R\setminus\{0\},\cdot),\; 
    (\R[>0], \cdot)$ topologische Gruppen bezüglich der Teilraumtopologie
    induziert von der Standardtopologie auf $\R^n$ bzw. $\R$. Alle drei sind
    lokalkompakt und hausdorffsch.

\item
    Bezüglich der Standardtopologie (d.\,h. der durch den Betrag induzierten
    Topologie) auf $\C$ ist der Einheitskreis 
    $\left\{ z\in\C \;\big\vert\; \abs{z}=1 \right\} \cong S^1 \subset \R^2$ mit
    der Multiplikation (in $\C$) eine kompakte topologische Gruppe.

\item
    Betrachtet man zu $n\in\N$ die Gruppe $\matGL(n,\R)$ der
    invertierbaren $(n\!\times\!n)$-Matrizen mit der von $\R^{n^2}$ induzierten
    Topologie, so ist diese bezüglich der Matrizenmultiplikation eine
    topologische Gruppe. Für die Untergruppen $\matSL(n,\R),\,\matO(n),\,\matSO(n)
    \leq \matGL(n,\R)$ gilt: Alle sind abgeschlossen in $\matGL(n,\R)$ und
    die letzteren beiden sogar kompakt.
    
    Diese Aussagen sieht man wie folgt ein:
    \vspace{-3pt} \[    % UGLY
        \det\colon\matGL(n,\R)\to\R^\times 
    \vspace{-3pt} \] 
    ist ein (surjektiver) stetiger Gruppenhomomorphismus und somit ist das
    Urbild der abgeschlossenen Untergruppe $\{1\} \subset \R^\times$ (also
    gerade $\matSL(n,\R)$) eine abgeschlossene Untergruppe von $\matGL(n,\R)$.
    Weiter ist $\matO(n)$ abgeschlossen als Urbild der abgeschlossenen Menge
    $\{0\} \subset \R^n$ unter der stetigen Abbildung $Q\mapsto Q\mt Q - E_n$
    (wobei $E_n$ die $(n\!\times\!n)$-Einheitsmatrix bezeichnet). Mit $\matSO(n)
    = \matSL(n,\R) \cap \matO(n)$ ist auch $\matSO(n)$ abgeschlossen.
    Mit dem Satz von Heine-Borel folgt dann die Kompaktheit, denn $\matO(n)$ ist
    außerdem beschränkt, da für $Q\in\matO(n)$ gilt: 
    \[ \opnorm{Q}_2 = \sup_{\norm{x}_2=1} \norm{Qx}_2 
        = \sup_{\norm{x}_2=1} \norm{x}_2 = 1
    . \]
    (Dabei bezeichnet $\opnorm{\,\cdot\,}_2$ die Operatornorm, induziert von der
    euklidschen Norm.)
    
\item
    Ein weiteres interessantes Beispiel ist die sogenannte
    \emph{Krulltopologie}, für welche wir nur kurz den Anwendungszweck
    beschreiben wollen: Hat man eine (nicht notwendigerweise endliche)
    galois'sche Körpererweiterung $L/K$, so kann man auf
    $\operatorname{Gal}(L/K)$ eine Topologie definieren, welche man
    \emph{Krulltopologie} (nach Wolfgang Krull) nennt. Mit Hilfe dieser
    topologischen Zusatzstruktur kann man dann den \emph{Hauptsatz der
    Galoistheorie} auf \emph{unendliche Galoiserweiterungen} übertragen, indem man
    \emph{abgeschlossene} Untergruppen der Galoisgruppe betrachtet.
    Eine Definiton der Krulltopologie und Einführung in die unendliche
    Galoistheorie kann in dem Artikel \enquote{Unendliche
    Galoistheorie}\cite{www:mp:unendlichegaloistheorie} nachgelesen werden.
\end{BspList}
%
Weitere Beispiele kann man Elstrod\cite[Kap.\,VIII,\,\S3,\,3.2]{bookc:elstrod11}
entnehmen.
%
Bis auf Weiteres bezeichnet $G$ ab jetzt immer eine beliebige topologische
Gruppe. Wir behandeln nun einige Eigenschaften topologischer Gruppen.
\begin{thLemma}%
    [Invertieren, Links- und Rechtsmultiplikation als Homöomorphismen]
    \label{tg:homoeom}
    \hfill\\
    %
    Sei $a\in G$. Die folgenden Abbildungen sind Homöomorphismen auf $G$:
    \begin{alignat*}{3}
        G &\to G,   &\quad     x &\mapsto x^{-1}
        & \qquad\quad &\text{(Invertieren)}
        \\
        G &\to G,   &\quad     x &\mapsto ax
        & &\text{(Linksmultiplikation mit $a$)}
        \\
        G &\to G,   &\quad     x &\mapsto xa
        & &\text{(Rechtsmultiplikation mit $a$)}
    \end{alignat*}
\end{thLemma}

\begin{proof}
    Nach Definition einer topologischen Gruppe ist Invertieren eine stetige
    Abbildung und offenbar ist diese zu sich selbst invers.
    \\
    Die Abbildung $\{a\}\times G\to G,\; (a,x)\mapsto ax$ ist stetig als
    Einschränkung der (nach Definiton) auf $G$ stetigen Multiplikation, womit
    auch
    \[ \bigl( x\mapsto (a,x) \bigr) \circ \bigl( (a,x) \mapsto ax \bigr) 
    \]
    als Komposition stetiger Abbildungen stetig ist. Außerdem ist die
    Linksmultiplikation mit~$a^{-1}$ offenbar eine Umkehrabbildung, welche nach
    denselben Argumenten stetig ist. 
    \\
    Analog für die Rechtsmultiplikation mit~$a$.
    \\
\end{proof}

Eine wichtige Folgerung daraus, welche wir oft verwenden werden, ohne es explizit
zu nennen, ist:
\begin{thKorollar}%
    [Translate offener/abgeschlossener Mengen, Produkte mit offenen Mengen]
    %
    Sei $a\in G$ und seien $U\subset G$ offen, $V\subset G$ abgeschlossen und
    $B\subset G$ beliebig.

    Dann sind auch $aU$ und $Ua$ offen sowie $aV$ und $Va$ abgeschlossen.
    Außerdem sind auch $B\mkern2muU$ und $UB$ offen.
\end{thKorollar}

\begin{proof}
    Die ersten beiden Aussagen folgen unmittelbar aus \cref{tg:homoeom} und
    der Tatsache, dass Homöomorphismen sowohl offen als auch abgeschlossen sind.
    Für die letzte Aussage schreiben wir
    \[ B\mkern2muU = \bigcup_{b\in B} bU \qqundqq UB = \bigcup_{b\in B} Ub, \]
    wodurch wir erkennen, dass $B\mkern2muU$ und $UB$ als Vereinigungen offener
    Mengen jeweils wieder offen sind. 
    \\
\end{proof}

Aus den letzten beiden Resultaten erhalten wir insbesondere, dass wir für
$a\in G$ und eine Umgebung $U\subset G$ von $e$ immer eine Umgebung $aU$ um $a$
mit gleichen topologischen Eigenschaften wie $U$ erhalten. Da Umgebungen um $e$
auch weiterhin eine große Rolle spielen werden, definieren wir:
\begin{thDef}[Umgebungen von $e$]
    Ist $G$ eine topologische Gruppe, so bezeichne
    \[ \frakU \defeq \{ U\subset G \Mid \text{$U$ ist Umgebung von $e$} \} \]
    die Menge aller Umgebungen von $e$.
\end{thDef}

Wir halten nun einige Aussagen über topologische Gruppen, welche wir immer
wieder benutzen werden, gemeinsam fest:
\begin{thLemma}[Allgemeine Aussagen über topologische Gruppen]\label{tg:basics}
    \hfill
    \begin{enumerate}[a)]
        \item \label{tg:basics:symV}
            Für alle $U\in\frakU$ existiert ein symmetrisches $V\in\frakU$ mit
            $V\subset U$.
        \item \label{tg:basics:VV}
            Für alle $U\in\frakU$ existiert ein $V\in\frakU$ mit $VV\subset U$.
        \item \label{tg:basics:symVV}
            Für alle $U\in\frakU$ existiert ein symmetrisches $V\in\frakU$ mit
            $VV\subset U$.
        \item \label{tg:basics:closureH}
            Ist $H\leq G$ eine Untergruppe, so auch ihr (topologischer) Abschluss
            $\setclosure{H}$.
        \item \label{tg:basics:openHclosed}
            Ist $H\leq G$ eine offene Untergruppe, so ist $H$ auch
            abgeschlossen.
        \item \label{tg:basics:KKcompact}
            Sind $K_1,K_2\subset G$ kompakt, so auch $K_1K_2$.
    \end{enumerate}
\end{thLemma}

\begin{proof}
    Seien $U\in\frakU$ und $H\leq G$ beliebig.
    \begin{enumerate}[(a)]
        \item
            Da $e^{-1} = e$ gilt und Invertieren stetig ist, ex. nach der
            lokalen Charakterisierung von Stetigkeit eine Umgebung $T\in\frakU$,
            so dass $T^{-1} \subset U$ gilt.
            Dann leistest $V \defeq T \cap T^{-1}$ offenbar das Gewünschte.
        
        \item
            Es gilt $e\mkern2mu e=e$ und weil die Multiplikation stetig ist, ex. wieder
            eine Umgebung $T\subset G\times G$ von $(e,e)\in G\times G$, so dass 
            %$\bigl( G\times G \to G,\; (a,b)\mapsto ab \bigr)(T) \subset U$ gilt.
            für alle $(t_1,t_2)\in T$ das Produkt $t_1t_2$ in $U$ liegt.
            Nach Definition der Produkttopologie gibt es also
            $T_1,T_2\in\frakU$, so dass $T_1\times T_2 \subset T$ erfüllt ist.
            Setzte dann $V\defeq T_1\cap T_2$, dann gilt offenbar $VV\subset U$.

        \item
            Wähle zunächst eine Umgebung $V'\in\frakU$ nach
            \ref{tg:basics:VV}. Wähle dann nach \ref{tg:basics:symV} eine
            symmetrische Umgebung $V\in\frakU$ mit $V\subset V'$. Dann gilt
            natürlich wegen $V\subset V'$: $VV \subset V'V' \subset U$.

        \item
            Wir sehen zunächst ein, dass die Abbildung
            \begin{align*}
                \phi\colon G\times G &\to G,    \\
                            (a,b)    &\mapsto ab^{-1}
            \end{align*}
            als Komposition stetiger Abbildungen auch stetig ist. Wir verwenden
            nun das bekannte Kriterium, dass $\setclosure H$ eine Untergruppe
            von $G$ ist, wenn für alle $x,y\in\setclosure{H}$ auch
            $xy^{-1}\in\setclosure{H}$ gilt.
            
            Seien nun also $x,y \in \setclosure{H}$ beliebig und sei $U\subset G$
            eine beliebige Umgebung von $xy^{-1}$. Weil obiges $\phi$ stetig
            ist und offenbar $\phi(x,y)=xy^{-1}$ gilt, existiert eine Umgebung
            $T\subset G$ von $(x,y)$ derart, dass $\phi(T) \subset U$ gilt. Nach
            Definition der Produkttopologie gibt es dann Umgebungen $T_1$ bzw.
            $T_2$ von $x$ bzw. $y$, so dass $(x,y) \in T_1\times T_2 \subset T$
            erfüllt ist. Wegen $x,y\in\setclosure{H}$ erhalten wir: 
            $T_1\cap H\neq\emptyset$ und $T_2\cap H\neq\emptyset$. Es folgt
            $\phi(T_1\times T_2)=T_1 T_2^{-1}\cap H\neq\emptyset$ und damit auch
            $U\cap H\neq\emptyset$, womit $xy^{-1}$ im Abschluss von $H$ liegt. 

        \item
            Sei $H$ zusätzlich offen. Wir betrachten nun die Linksnebenklassen
            $gH$ für $g\in G$. Da $H$ offen ist, ist auch $gH$ offen für alle
            $g\in G$ und bekannterweise kann $G$ als disjunkte Vereinigung aller
            Linksnebenklassen geschrieben werden. Damit gilt:
            \[ \bigdotcup_{x\in G\setminus\{e\}} xH \]
            ist offen als Vereinigung offener Mengen und somit ist das
            Komplement in $G$, also gerade $H$, abgeschlossen.

        \item
            Für kompakte Teilmengen $K_1,K_2\subset G$ ist auch $K_1\times K_2$
            in $G\times G$ bezüglich der Produkttopologie kompakt und damit
            folgt die Kompaktheit von $K_1K_2$ daraus, dass dies das Bild von
            $K_1\times K_2$ unter der stetigen Gruppenmultiplikation ist.
    \end{enumerate}
\end{proof}


%\begin{thDef}(Links-/Rechtstranslation)
%    Sei $G$ eine topologische Gruppe und $f$ eine auf $G$ definierte Funktion.
%\end{thDef}<++>





































\chapter{Existenz und Eindeutigkeit des Haar'schen Maßes}
Wir wollen nun die Existenz und Eindeutigkeit (bis auf einen positiven Faktor)
eines Haar-Maßes auf LKH-Gruppen zeigen. 
% (Im Folgenden meinen wir mit \enquote{Eindeutigkeit des Haar-Maßes} 
% immer die Eindeutigkeit bis auf einen konstanten Faktor.)
Für den Existenzbeweis halten wir uns an den erstmals 1940 von 
\emph{Andr\'e Weil} in allgemeiner Form veröffentlichten Beweis für beliebige
LKH-Gruppen, wie man ihn etwa bei
Elstrod\cite[Kap.\,VIII,\;\S3,\;3.12]{bookc:elstrod11} 
oder auch bei Folland\cite[\S11,\;11.8]{bookc:folland99} findet.
Dieser benutzt das \emph{Auswahlaxiom} in Form des
\emph{Satzes von Tychonoff}, weshalb es an dieser Stelle erwähnenswert ist, dass
\emph{Henri Cartan} (ebenfalls 1940) auch einen Beweis gefunden hat, der ohne
selbiges auskommt. (Siehe dazu auch: Alfsen\cite{artcle:alfsen63}.)
Für den Eindeutigkeitsbeweis verwenden wir ein Argument, das hauptsächlich auf
dem Gebrauch des \emph{Satzes von Fubini} beruht, siehe beispielsweise
Folland\cite[\S11,\;11.9]{bookc:folland99}. Auch hier gibt es alternative
Beweise, die mit elementareren Mitteln auskommen. Der ursprüngliche Beweis von
\emph{A. Weil} macht zum Beispiel keinen Gebrauch vom Satz von Fubini und ein
Beweis nach diesem Schema findet man im oben schon zitierten Satz bei Elstrod.

\medskip
Da wir schon gezeigt haben, dass sich Haar-Maße und Haar-Integrale entsprechen,
genügt es, die Existenz und (im Wesentlichen) Eindeutigkeit eines Haar-Integrals
zu zeigen, und dies ist auch der übliche Ansatz. Im Übrigen meinen wir hier und
im Rest dieses Kapitels stets \emph{linkes} Haar-Maß und \emph{linkes}
Haar-Integral, wenn wir es nicht explizit spezifizieren. Um nun eine Motivation
für die weiter unten auftrende Konstruktion zu geben, betrachten wir folgendes
Szenario: Sei $G$ eine LKH-Gruppe und seien $K\subset G$ eine kompakte und
$U\in\frakU$ eine offene Menge. Weil wir ein von innen reguläres Maß suchen,
genügt es prinzipiell, das Maß auf kompakten Mengen zu kennen. Wir möchten nun
also ein \enquote{Maß} für $K$ angeben, was wir zum Beispiel unter Verwendung
von $U$ wie folgt tun können:
Weil $K$ kompakt ist, finden wir ein $n\in\N$ und endlich viele $x_i\in G$, so
dass $(x_iU)_{i\in\{1,\ldots,n\}}$ eine Überdeckung von~$K$ bildet. Definiert
man nun $(K : U)$ als das minimale derartige $n\in\N$, so misst $(K : U)$ in
gewissem Rahmen wie groß $K$ in Relation zu $U$ ist. Die Idee ist nun, $U$ immer
kleiner zu wählen, so dass $U$ auf $\{e\}$ zusammenschrumpft und man somit die
Menge $K$ mit immer weniger \enquote{Überlappungen} überdecken kann. Mittels
einer Normierung, also $(K : U)/(K_0 : U)$ für festes $K_0\subset G$, können wir
erhoffen, dass sich bei dem eben skizzierten Grenzprozess ein Wert für diesen
Quotienten einstellt und wir dadurch ein (offenbar linksinvariantes) Maß auf~$G$
bekommen.




















\chapter{Anwendung: Die modulare Funktion}
Wir wollen nun eine erste Schlussfolgerung aus der soeben bewiesenen
Eindeutigkeit des Haar-Maßes ziehen. Sei dazu erneut $G$ eine LKH-Gruppe und sei
$\mu$ ein Haar-Maß auf $G$.

\begin{thLemma}%
    [Bildmaß eines linksinvarianten Maßes unter Rechtsmultiplikation]
    \label{mod:pushforwardmeasure}
    \hfill\\
    %
    Sei $y\in G$ und $\nu$ ein linksinvariantes Maß auf $G$.
    Dann ist auch das von der Rechtsmultiplikation mit $y^{-1}$ induzierte Maß
    $\nu \circ (x\mapsto xy)$ linksinvariant.
\end{thLemma}

\begin{proof}
    Nach dem Assiziativgesetz gilt für alle $B\in\borelsigmaalg(G)$ und $z\in G$:
    \[ \bigl(\nu \circ (x\mapsto xy)\bigr)(zB) 
        = \nu\bigl( (zB) y \bigr) 
        = \nu\bigl( z (By) \bigr)
        = \nu(By) = \bigl(\nu \circ (x\mapsto xy)\bigr)(B)
    \]
\end{proof}

\begin{thKorollar}%
    [Haar-Maße unter Rechtsmultiplikation]
    \label{mod:pushforwardhaar}
    %
    Für das Haar-Maß $\mu$ ist auch $\mu_y\colon B\mapsto\mu(By)$ ein Haar-Maß
    und es gibt ein $\Delta(y)\in\R[>0]$, so dass $\mu_y = \Delta(y)\,\mu$ gilt.
    Außerdem ist $\Delta(y)$ unabhängig von der Wahl von $\mu$.
\end{thKorollar}

\begin{proof}
    Die erste Aussage folgt unmittelbar aus \cref{mod:pushforwardmeasure} (und
    der Tatsache, dass $\mu_y$ auch von innen regulär ist, analog wie im Beweis
    von \cref{tg:rmeasuresVSfunctionals}) und
    die zweite aus \cref{pf:uniqueness}. Aus letzterem folgt außerdem die letzte
    Behauptung, denn: Ist $\nu$ ein weiteres Haar-Maß, so gilt $\mu = c\,\nu$ für
    ein $c\in\R[>0]$ und damit ergibt sich: 
    $\Delta(y)\,c\,\nu = \Delta(y)\,\mu = \mu_y = (c\,\nu)_y = c \, \nu_y$,
    woraus $\Delta(y)\,\nu = \nu_y$ folgt.
    \\
\end{proof}

Da wir mit \cref{mod:pushforwardhaar} gezeigt haben, dass $\Delta$ unabhängig
von der Wahl des Haar-Maßes ist, gibt dies Anlass zu folgender Definition:

\begin{thDef}[Modulare Funktion, unimodular]
    Die Abbildung $\Delta\colon G\to\R[>0]$ aus \cref{mod:pushforwardhaar}
    heißt \emph{modulare Funktion (von $G$)}.
    Gilt $\Delta \equiv 1$, so nennen wir $G$ \emph{unimodular}.
\end{thDef}

\begin{BspList}[\label{mod:unimodbsp}]
\item
    Jede abelsche LKH-Gruppe ist unimodular.
    
\item\label{mod:unimodbsp:kompakt}
    Ist die LKH-Gruppe $G$ kompakt, so ist sie unimodular. Begründung: Weil $G$
    kompakt ist, ist das Maß über den gesamten Raum endlich und es gilt dann: 
    \[ \infty > \Delta(y)\,\mu(G) = \mu(Gy) = \mu(G) . \]
    Wir dividieren durch $\mu(G)$ und erhalten $\Delta(y)=1$ für alle $y\in G$.
\end{BspList}

Zuletzt wollen wir noch zeigen, dass $\Delta\colon G \to \R[>0]$ sogar einige
schöne Eigenschaften besitzt, konkret:

\begin{thSatz}[Modulare Funktion als stetiger Gruppenhomomorphismus]
    Die modulare Funktion $\Delta$ ist ein stetiger Gruppenhomomorphismus
    von $G$ nach $(\R[>0],\,\cdot\,)$.
\end{thSatz}

\begin{proof}
    Seien $x,z\in G$ beliebig und sei $B\in\borelsigmaalg(G)$ kompakt. Dann gilt
    nach Definition der modularen Funktion:
    \[ \Delta(xz) \, \mu(B) = \mu(Bxz) = \mu\bigl( (Bx) z \bigr)
        = \Delta(z) \, \mu(Bx) = \Delta(z) \Delta(x) \mu(B)
    \]
    Da $\R[>0]$ abelsch ist, ist $\Delta$ also in der Tat ein Homomorphismus.
    Für den zweiten Teil gehen wir wie folgt vor: Wir zeigen die Stetigkeit von
    $\Delta$ im Punkt~$e$ und wie man sich leicht überlegt, folt aus der
    Stetigkeit eines Gruppenhomomorphismus im neutralen Element (oder einem
    beliebigen anderen Punkt) bereits die Stetigkeit auf der gesamten Gruppe.
    
    Zunächst nutzen wir aus, das nach dem Transformationssatz \pref{tg:trafo}
    für alle $y\in G$ und alle integrierbaren Funktionen $f$ auf $G$ gilt:
    \[ \int_G R_yf \dif\mu = \int_G f \dif{\mu_{y^{-1}}
        = \Delta(y^{-1}) \int_G f \dif\mu
    \]
    (Dabei sei $\mu_{y^{-1}}\colon B \mapsto \mu(By^{-1})$ wie in
    \cref{mod:pushforwardhaar}.)
    Wir können also auch zeigen, dass die linke Seite als Abbildung in~$y$
    stetig bei $e$ ist, denn dann muss dies für die rechte Seite und
    insbesondere für $\bigl(y\mapsto\Delta(y^{-1}})\bigr)\circ\bigl(y\mapsto
    y^{-1}\bigr) = \Delta$ auch gelten. Wähle dazu ein $f\in\contcompplus$ mit
    $\int_G f \dif\mu = 1$ und definiere
    \[ I\colon G\to\R[>0],\quad y\mapsto \int_G R_yf . \]
    Wegen $I(e)=1$ müssen wir also für jedes $\epsilon\in\R[>0]$ eine
    Umgebung $U\in\frakU$ um $e$ finden, so dass
    $I(U)\subset (1-\epsilon,1+\epsilon)$ oder äquivalent
    $\forall\,u\in U\colon \abs{I(u)-1} < \epsilon$ erfüllt ist.
    
    Wir setzen $K\defeq\supp(f)$ und wählen $\epsilon\in\R[>0]$ beliebig.
    Weil $G$ lokalkompakt ist, gibt es eine kompakte Umgebung $K'\in\frakU$ 
    um~$e$ und wir setzen $K''\defeq KK'$. Nach
    \mycref{tg:basics:KKcompact} ist dann $K''$ eine kompakte Menge und es gilt
    insbesondere $\mu(K'')<\infty$.
    Da $f$ nach \cref{tg:unicont} rechts-gleichmäßig stetig ist, existiert
    eine Umgebung $U'\in\frakU$ von~$e$, so dass für $y\in U'$ gilt:
    \[ \supnorm{f - R_yf} < \frac{\epsilon}{\mu(K'')}  \] 
    Setzte nun $U\defeq U'\cap K'$, was immer noch eine Umgebung um~$e$ ist.
    Dann gilt für alle $y\in U$:
    \[ \supp(f - R_yf) \subset K \cup KK' = K'' \]
    Nun können wir für $y\in U$ folgendermaßen abschätzen:
    \begin{align*}
        \abs{I(y)-1}                                            %
        &= \abs*{                                               %
            \int_G (R_yf - f) \dif\mu                           %
        }                                                       \\
        &\leq \int_G \abs{ f - R_yf } \dif\mu                   \\
        &\leq \frac{\epsilon}{\mu(K'')}\;\mu(K'') = \epsilon    %%
    \end{align*}
    Also ist $U$ die gesuchte Umgebung und wir sind fertig.
    \\
\end{proof}

Wenn man nun weiß, dass $\Delta$ ein stetiger Gruppenhomomorphismus ist, so kann
man für\ref{mod:unimodbsp:kompakt} von \cref{mod:unimodbsp} noch einen weiteren
schönen Beweis geben:\\
Ist $G$ kompakt, so muss $\Delta(G)\subset\R[>0]$ eine kompakte Untergruppe von
$(\R[>0],\,\cdot\,)$ sein. Wie man leicht zeigt, ist aber $\{1\}\leq\R[>0]$ die
einzige solche Untergruppe, womit $G$ schon unimodular sein muss.



















\nocite{bookc:folland95}
\nocite{www:mp:gruppenzwang7}

\appendix
\bibliographystyle{plaindin}
\bibliography{bibsources}

\end{document}






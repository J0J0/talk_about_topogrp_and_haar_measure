\documentclass[11pt,a4paper,ngerman,DIV=11,bibliography=totoc]{scrreprt}

%%%%%%%%%%%%%%%%%%%%%%%%%%%%%%%%%%%%%%%%%%%%%%%%%%%%%%%%%%%%%%%%%%%%%
%%% packages
%%%%%%%%%%%%%%%%%%%%%%%%%%%%%%%%%%%%%%%%%%%%%%%%%%%%%%%%%%%%%%%%%%%%%

\usepackage[utf8]{inputenc}
\usepackage[T1]{fontenc}
\usepackage[ngerman]{babel}

\usepackage{amsmath}
\usepackage{amssymb}
\usepackage{amsthm}
\usepackage{mathtools}

\usepackage[babel]{csquotes}
\usepackage[shortlabels]{enumitem}
\usepackage[numbers,sort&compress]{natbib}
\usepackage{ifmtarg}
\usepackage{xstring}
\usepackage{remreset}

%\usepackage[pdftex]{graphicx}
%\usepackage[all]{xy}

\usepackage[pdftex,bookmarks,colorlinks=false,pdfborder={0 0 0},%
            pdfauthor={Johannes Prem}]{hyperref}
%
\usepackage{cleveref}

\usepackage{helpers} % my own helpers.sty

% 'ß' in the title requires this to be called outside \usepackage's options
\hypersetup{%
    pdftitle={Seminar Maßtheorie - Vortrag 11 und 12: Topologische
              Gruppen und das Haar'sche Maß}}

%%%%%%%%%%%%%%%%%%%%%%%%%%%%%%%%%%%%%%%%%%%%%%%%%%%%%%%%%%%%%%%%%%%%%
%%% macro definitions and other things
%%%%%%%%%%%%%%%%%%%%%%%%%%%%%%%%%%%%%%%%%%%%%%%%%%%%%%%%%%%%%%%%%%%%%

% global redefinition of paragraph spacing
%\setlength{\parindent}{0pt}
%\setlength{\parskip}{0.5em}

% don't reset footnote numbers
\makeatletter
\@removefromreset{footnote}{chapter}
\makeatother


% make parenthesized versions of \ref and cleveref's \cref
\newcommand*{\pref}[1]{(\ref{#1})}
\newcommand*{\pcref}[1]{(\cref{#1})}

% make a even more clever \mycref that produces "Lemma 42a)" etc.
\newcommand{\mycref}[1]{%
    \begingroup%
    \StrCount{#1}{:}[\mycrefCount]%
    \StrBefore[\mycrefCount]{#1}{:}[\myrefMain]%
    \expandafter\cref\expandafter{\myrefMain}\,\ref{#1}%
    \endgroup%
}

% make \varepsilon and \varphi default
\varifygreekletters{\epsilon\phi}

% change the qedsymbol to my favoured blacksquare
\renewcommand{\qedsymbol}{$\blacksquare$}

% style for /all/ theorem like environments
\newtheoremstyle{mythms}
 {15pt}% space above
 {12pt}% space below 
 {}% body font
 {}% indent amount
 {\bfseries}% theorem head font
 {.}% punctuation after theorem head
 {0.6cm plus 0.25cm minus 0.1cm}% space after theorem head (\newline possible)
 {}% theorem head spec 
 
% set style and define thm like environments
\theoremstyle{mythms}
\newtheorem{globalnum}{DUMMY DUMMY DUMMY}[chapter]
\newtheorem{thDef}[globalnum]{Definition}
\newtheorem{thNotation}[globalnum]{Notation}
\newtheorem{thSatz}[globalnum]{Satz}
%\newtheorem{thPropos}[globalnum]{Proposition}
\newtheorem{thLemma}[globalnum]{Lemma}
\newtheorem{thKorollar}[globalnum]{Korollar}

\newtheorem{thBemerkung}[globalnum]{Bemerkung}
\newtheorem{thBeisp}[globalnum]{Beispiel}
\newtheorem{thBeispiele}[globalnum]{Beispiele}
\newenvironment{BspList}[1][]{%
\nopagebreak\begin{thBeispiele}#1%
\hfill\begin{enumerate}[a),parsep=0pt,itemsep=0.8ex,leftmargin=2em]%
}{%
\end{enumerate}\end{thBeispiele}
}
%

% also define a 'proofsketch' version of 'proof'
\newenvironment{proofsketch}[1][]{%
\begin{proof}[Beweisskizze#1]
}{%
\end{proof}
}

% inject pdfbookmarks at thm like environments
\makeatletter
\let\origthmhead=\thmhead
\renewcommand{\thmhead}[3]{%
\origthmhead{#1}{#2}{#3}%
\pdfbookmark[2]{#1\@ifnotempty{#1}{ }#2\thmnote{ (#3)}}{#1#2}%
}
\makeatother

% new math operators
\DeclareMathOperator*{\bigdotcup}{\overset{\mkern0mu\scalebox{0.6}{$\bullet$}}{\bigcup}}

% new math 'operators'
\DeclareMathOperator{\matGL}{GL}
\DeclareMathOperator{\matSL}{SL}
\DeclareMathOperator{\matO}{O}
\DeclareMathOperator{\matSO}{SO}
\DeclareMathOperator{\supp}{supp}
\DeclareMathOperator{\id}{id}
\DeclareMathOperator{\J}{J\mkern-1mu}
\DeclareMathOperator{\Potmenge}{\mathcal{P}}
%\DeclareMathOperator{\Kern}{ker}
%\DeclareMathOperator{\Image}{im}

% make quantors that use \limits per default
\DeclareMathOperator*{\Exists}{\exists}
\DeclareMathOperator*{\forAll}{\forall}

% define an 'abs', 'norm' and 'Spann' command
\DeclarePairedDelimiter{\abs}{\lvert}{\rvert}
\DeclarePairedDelimiter{\norm}{\lVert}{\rVert}
\DeclarePairedDelimiter{\Spann}{\langle}{\rangle}

%---
% the following creates an operator norm as a tripple stroke \vert
% (source: mathabx fonts)
\DeclareFontFamily{U}{matha}{\hyphenchar\font45}
\DeclareFontShape{U}{matha}{m}{n}{
      <5> <6> <7> <8> <9> <10> gen * matha
      <10.95> matha10 <12> <14.4> <17.28> <20.74> <24.88> matha12
      }{}
\DeclareSymbolFont{matha}{U}{matha}{m}{n}
\DeclareFontFamily{U}{mathx}{\hyphenchar\font45}
\DeclareFontShape{U}{mathx}{m}{n}{
      <5> <6> <7> <8> <9> <10>
      <10.95> <12> <14.4> <17.28> <20.74> <24.88>
      mathx10
      }{}
\DeclareSymbolFont{mathx}{U}{mathx}{m}{n}

\DeclareMathDelimiter{\vvvert}{0}{matha}{"7E}{mathx}{"17}
\DeclarePairedDelimiter{\opnorm}{\vvvert}{\vvvert}
%---

% define missing arrows
\newcommand{\longto}{\longrightarrow}
\newcommand{\longhookrightarrow}{\lhook\joinrel\relbar\joinrel\rightarrow}

% provide mathbb symbols \N \Z \Q \R and \C
\defmathbbsymbols{N Z Q C}
\defmathbbsymbolsubs{R}

% define some set specific macros
\newcommand{\setclosure}[1]{\overline{#1}}
\newcommand{\setinterior}[1]{#1^\circ}
\newcommand{\setboundary}[1]{\partial #1}

% just some shortcuts
\newcommand{\mr}{\mathrm}
\newcommand{\mt}{^\mathsf{t}}
\newcommand{\defeq}{\coloneqq}
\newcommand{\eqdef}{\eqqcolon}
%\newcommand{\barfrak}[1]{\bar{\mathfrak{#1}}}
%\newcommand{\len}[1][R]{\modulelength_{#1}}
%\newcommand{\ZRest}[1]{\Z/#1\Z}
%\newcommand{\D}{\mr{d}}
%\newcommand{\A}[2]{a_{#1#2}}
%\newcommand{\vzfix}{\phantom{-}}
\newcommand{\qiffq}{\quad\iff\quad}
\newcommand{\qimpliesq}{\quad\implies\quad}
\newcommand{\qoderq}{\qtextq{oder}}
\newcommand{\qqundqq}{\qqtextqq{und}}
\newcommand{\qundq}{\qtextq{und}}
\newcommand{\qqtextqq}[1]{\qquad\text{#1}\qquad}
\newcommand{\qtextq}[1]{\quad\text{#1}\quad}
\newcommand{\supnorm}[1]{\norm{#1}_\infty}
\newcommand{\borelsigmaalg}{\mathcal{B}}
\newcommand{\dif}[2][\;]{#1\mathrm{d} #2}
\newcommand{\half}{\frac{1}{2}}
\newcommand{\thalf}{\tfrac{1}{2}}
\newcommand{\Nfolge}[1]{\left(#1_n\right)_{n\in\N}}
\newcommand{\pot}[1]{\Potmenge(#1)}


\newcommand{\frakU}{\mathfrak{U}}

\newcommand{\continuous}{C}
\newcommand{\contcomp}{C_c}
\newcommand{\contcompplus}{C_c^+}

\DeclareMathOperator{\Realteil}{Re}
\DeclareMathOperator{\Imaginaerteil}{Im}
\let\Re=\Realteil
\let\Im=\Imaginaerteil


% make a \Mid macro as flexible replacement for \mid in sets definitions
\newcommand{\Mid}[1][\,]{%
#1%
\ifnum\currentgrouptype=16%
\middle\vert\else\vert\fi%
#1%
}
% \dots and a version for custom size control
\newcommand{\cMid}[2][\,]{%
#1#2\vert#1%
}

% listing with -- is nicer than with bullets 
\setlist[itemize,1]{label=--}

%% xy tip selection (ComputerModern)
%%\SelectTips{cm}{}
%%\UseTips

% start at chapter 0
\setcounter{chapter}{-1}

%%%%%%%%%%%%%%%%%%%%%%%%%%%%%%%%%%%%%%%%%%%%%%%%%%%%%%%%%%%%%%%%%%%%%
%%% document
%%%%%%%%%%%%%%%%%%%%%%%%%%%%%%%%%%%%%%%%%%%%%%%%%%%%%%%%%%%%%%%%%%%%%

\begin{document}


\subject{Seminar: Maßtheorie}
\title{Topologische Gruppen\\ und Haar'sches Maß}
\author{Johannes Prem}
\date{05.04.2013}

\maketitle


\chapter{Vorwort, Notation und Konventionen}
Dieses Skript behandelt \emph{Haar'sche Maße} auf \emph{topologischen Gruppen}.
Dazu wird zunächst der letztere Begriff eingeführt und anhand einiger Beispiele
verdeutlicht. Dann werden einige einfache Resultate gezeigt und schließlich der
Begriff des \emph{Haar-Maßes} eingeführt. Zuletzt wird die Existenz und
(im Wesentlichen) Eindeutigkeit des Haar-Maßes auf \emph{lokalkompakten
hausdorffschen topologischen Gruppen} gezeigt und als erste Anwendung die
\emph{modulare Funktion} besprochen.


\bigskip
In diesem Skript wird folgende Notation verwendet:
\begin{itemize}
    \item
        In Analogie zu der in englischer Literatur manchmal zu findenden
        Abkürzung \enquote{LCH space} werden wir für eine topologische Gruppe,
        die außerdem lokalkompakt und hausdorffsch ist, die Abkürzung
        \emph{LKH-Gruppe} verwenden.
        
    \item
        Sowohl $\subset$ als auch $\subseteq$ stehen für: enthalten oder gleich.
        Echt enthalten wird durch $\subsetneq$ gekennzeichnet.
    
    \item
        Die \emph{Natürlichen Zahlen $\N$} beginnen mit $0$.
    
    %\item
    %    Zu einem Ring $R$ bezeichnet $R^\times$ die Einheitengruppe des Rings.
\end{itemize}


\bigskip
Weiterhin vereinbaren wir für einige Begriffe, die in der Literatur
unterschiedlich definiert werden, Folgendes% (wobei $(X,\Topo)$ ein beliebiger topologischer Raum sei)
:
\begin{itemize}
    \item 
        Ein topologischer Raum ist \emph{lokalkompakt} genau dann,
        wenn jeder Punkt eine kompakte Umgebung besitzt.
        
    \item
        Ein topologischer Raum ist \emph{regulär} genau dann,
        wenn er die Trennungsaxiome $T_2$ und $T_3$ erfüllt, d.\,h. wenn er ein
        Hausdorff-Raum ist und je eine (nicht-leere) abgeschlossene Menge und 
        ein Punkt aus dem Komplement selbiger durch (offene) Umgebungen
        getrennt werden können. (Man zeigt leicht, dass es schon genügt, 
        statt $T_2$ nur $T_0$ zu fordern, um eine äquivalente Definition zu
        erhalten.)
\end{itemize}

\chapter{Topologische Gruppen}
\section{Definition und Eigenschaften}
Für eine Gruppe $G$ bezeichne stets $e$ das neutrale Element. Weiter werden wir
im Folgenden die auftretenden Gruppen meist multiplikativ schreiben und dabei
wie üblich das explizite Multiplikationszeichen unterdrücken. Das Inverse zu
einem Element $a\in G$ bezeichnen wir mit $a^{-1}$.

\begin{thDef}[Topologische Gruppe]
    Eine Gruppe $(G,\cdot, e)$ ist eine \emph{topologische Gruppe}, wenn auf ihr
    zusätzlich eine Topologie erklärt ist, für welche die Gruppenoperationen
    \begin{alignat*}{2}
        G \times G &\to G,  &\quad (a,b) &\mapsto ab
        \\
        G &\to G,           &\quad     a &\mapsto a^{-1}
    \end{alignat*}
    stetig sind. Dabei wird $G\times G$ mit der Produkttopologie versehen.
\end{thDef}

Sprechen wir im Folgenden davon, dass eine topologische Gruppe irgendeine
topologische Eigenschaft besitzt, so beziehen wir diese immer auf den
unterliegenden topologischen Raum.
Ist $G$ eine Gruppe und sind $A,B\subset G$ Teilmengen von $G$ sowie $x\in G$
ein Element von $G$, so benutzen wir außerdem folgende (intuitive) Notationen:
\begin{thNotation} 
    \begin{align*}
        xA &\defeq \{ xa \Mid a\in A \},            &
        Ax &\defeq \{ ax \Mid a\in A \},            \\
        AB &\defeq \{ ab \Mid a\in A,\, b\in B \},  &
        A^{-1} &\defeq \{ a^{-1} \Mid a\in A \}     %
    \end{align*}
\end{thNotation}

Gilt für eine Teilmenge $A\subset G$ die Gleichheit $A=A^{-1}$, so nennen wir
$A$~\emph{symmetrisch}. Nun geben wir aber zunächst einmal ein paar Beispiel an:
\begin{BspList}
\item
    Eine beliebige Gruppe $G$ ist bezüglich der diskreten Topologie eine
    topologische Gruppe. Diese ist dann trivialerweise eine LKH-Gruppe.
    
\item
    Es sind (für $n\in\N$) $(\R^n,+),\; (\R\setminus\{0\},\cdot),\; 
    (\R[>0], \cdot)$ topologische Gruppen bezüglich der Teilraumtopologie
    induziert von der Standardtopologie auf $\R^n$ bzw. $\R$. Alle drei sind
    lokalkompakt und hausdorffsch.

\item
    Bezüglich der Standardtopologie (d.\,h. der durch den Betrag induzierten
    Topologie) auf $\C$ ist der Einheitskreis 
    $\left\{ z\in\C \;\big\vert\; \abs{z}=1 \right\} \cong S^1 \subset \R^2$ mit
    der Multiplikation (in $\C$) eine kompakte topologische Gruppe.

\item
    Betrachtet man zu $n\in\N$ die Gruppe $\matGL(n,\R)$ der
    invertierbaren $(n\!\times\!n)$-Matrizen mit der von $\R^{n^2}$ induzierten
    Topologie, so ist diese bezüglich der Matrizenmultiplikation eine
    topologische Gruppe. Für die Untergruppen $\matSL(n,\R),\,\matO(n),\,\matSO(n)
    \leq \matGL(n,\R)$ gilt: Alle sind abgeschlossen in $\matGL(n,\R)$ und
    die letzteren beiden sogar kompakt.
    
    Diese Aussagen sieht man wie folgt ein:
    \[ \det\colon\matGL(n,\R)\to\R^\times 
    \] 
    ist ein (surjektiver) stetiger Gruppenhomomorphismus und somit ist das
    Urbild der abgeschlossenen Untergruppe $\{1\} \subset \R^\times$ (also
    gerade $\matSL(n,\R)$) eine abgeschlossene Untergruppe von $\matGL(n,\R)$.
    Weiter ist $\matO(n)$ abgeschlossen als Urbild der abgeschlossenen Menge
    $\{0\} \subset \R^{n\times n}$ unter der stetigen Abbildung $Q\mapsto Q\mt Q - E_n$
    (wobei $E_n$ die $(n\!\times\!n)$-Einheitsmatrix bezeichnet). Mit $\matSO(n)
    = \matSL(n,\R) \cap \matO(n)$ ist auch $\matSO(n)$ abgeschlossen.
    Mit dem Satz von Heine-Borel folgt dann die Kompaktheit, denn $\matO(n)$ ist
    außerdem beschränkt, da für $Q\in\matO(n)$ gilt: 
    \[ \opnorm{Q}_2 = \sup_{\norm{x}_2=1} \norm{Qx}_2 
        = \sup_{\norm{x}_2=1} \norm{x}_2 = 1
    . \]
    (Dabei bezeichnet $\opnorm{\,\cdot\,}_2$ die Operatornorm, induziert von der
    euklidschen Norm.)
    
\item
    Ein weiteres interessantes Beispiel ist die sogenannte
    \emph{Krulltopologie}, für welche wir nur kurz den Anwendungszweck
    beschreiben wollen: Hat man eine (nicht notwendigerweise endliche)
    galois'sche Körpererweiterung $L/K$, so kann man auf
    $\operatorname{Gal}(L/K)$ eine Topologie definieren, welche man
    \emph{Krulltopologie} (nach Wolfgang Krull) nennt. Mit Hilfe dieser
    topologischen Zusatzstruktur kann man dann den \emph{Hauptsatz der
    Galoistheorie} auf \emph{unendliche Galoiserweiterungen} übertragen, indem man
    \emph{abgeschlossene} Untergruppen der Galoisgruppe betrachtet.
    Eine Definiton der Krulltopologie und Einführung in die unendliche
    Galoistheorie kann in dem Artikel \enquote{Unendliche
    Galoistheorie}\cite{www:mp:unendlichegaloistheorie} nachgelesen werden.
\end{BspList}
%
Weitere Beispiele kann man Elstrod\cite[Kap.\,VIII,\,\S3,\,3.2]{bookc:elstrod11}
entnehmen.
%
Bis auf Weiteres bezeichnet $G$ ab jetzt immer eine beliebige topologische
Gruppe. Wir behandeln nun einige Eigenschaften topologischer Gruppen.
\begin{thLemma}%
    [Invertieren, Links- und Rechtsmultiplikation als Homöomorphismen]
    \label{tg:homoeom}
    \hfill\\
    %
    Sei $a\in G$. Die folgenden Abbildungen sind Homöomorphismen auf $G$:
    \begin{alignat*}{3}
        G &\to G,   &\quad     x &\mapsto x^{-1}
        & \qquad\quad &\text{(Invertieren)}
        \\
        G &\to G,   &\quad     x &\mapsto ax
        & &\text{(Linksmultiplikation mit $a$)}
        \\
        G &\to G,   &\quad     x &\mapsto xa
        & &\text{(Rechtsmultiplikation mit $a$)}
    \end{alignat*}
\end{thLemma}

\begin{proof}
    Nach Definition einer topologischen Gruppe ist Invertieren eine stetige
    Abbildung und offenbar ist diese zu sich selbst invers.
    \\
    Die Abbildung $\{a\}\times G\to G,\; (a,x)\mapsto ax$ ist stetig als
    Einschränkung der (nach Definiton) auf $G$ stetigen Multiplikation, womit
    auch
    \[ \bigl( x\mapsto (a,x) \bigr) \circ \bigl( (a,x) \mapsto ax \bigr) 
    \]
    als Komposition stetiger Abbildungen stetig ist. Außerdem ist offenbar die
    Linksmultiplikation mit $a^{-1}$ eine Umkehrabbildung, welche nach
    denselben Argumenten stetig ist. 
    \\
    Analog für die Rechtsmultiplikation mit~$a$.
    \\
\end{proof}

Eine wichtige Folgerung daraus, welche wir oft verwenden werden, ohne es explizit
zu nennen, ist:
\begin{thKorollar}%
    [Translate offener/abgeschlossener Mengen, Produkte mit offenen Mengen]
    \label{tg:translates}
    %
    Sei $a\in G$ und seien $U\subset G$ offen, $V\subset G$ abgeschlossen und
    $B\subset G$ beliebig.

    Dann sind auch $aU$ und $Ua$ offen sowie $aV$ und $Va$ abgeschlossen.
    Außerdem sind auch $B\mkern2muU$ und $UB$ offen.
\end{thKorollar}

\begin{proof}
    Die ersten beiden Aussagen folgen unmittelbar aus \cref{tg:homoeom} und
    der Tatsache, dass Homöomorphismen sowohl offen als auch abgeschlossen sind.
    Für die letzte Aussage schreiben wir
    \[ B\mkern2muU = \bigcup_{b\in B} bU \qqundqq UB = \bigcup_{b\in B} Ub, \]
    wodurch wir erkennen, dass $B\mkern2muU$ und $UB$ als Vereinigungen offener
    Mengen jeweils wieder offen sind. 
    \\
\end{proof}

Aus den letzten beiden Resultaten erhalten wir insbesondere, dass wir für
$a\in G$ und eine Umgebung $U\subset G$ von $e$ immer eine Umgebung $aU$ um $a$
mit gleichen topologischen Eigenschaften wie $U$ erhalten. Da Umgebungen um $e$
auch weiterhin eine große Rolle spielen werden, definieren wir:
\begin{thDef}[Umgebungen von \texorpdfstring{$e$}{e}]
    Ist $G$ eine topologische Gruppe, so bezeichne
    \[ \frakU \defeq \{ U\subset G \Mid \text{$U$ ist Umgebung von $e$} \} \]
    die Menge aller Umgebungen von $e$.
\end{thDef}

Wir halten nun einige Aussagen über topologische Gruppen, welche wir immer
wieder benutzen werden, gemeinsam fest:
\begin{thLemma}[Allgemeine Aussagen über topologische Gruppen]\label{tg:basics}
    \hfill
    \begin{enumerate}[a)]
        \item \label{tg:basics:symV}
            Für alle $U\in\frakU$ existiert ein symmetrisches $V\in\frakU$ mit
            $V\subset U$.
        \item \label{tg:basics:VV}
            Für alle $U\in\frakU$ existiert ein $V\in\frakU$ mit $VV\subset U$.
        \item \label{tg:basics:symVV}
            Für alle $U\in\frakU$ existiert ein symmetrisches $V\in\frakU$ mit
            $VV\subset U$.
        \item \label{tg:basics:closureH}
            Ist $H\leq G$ eine Untergruppe, so auch ihr (topologischer) Abschluss
            $\setclosure{H}$.
        \item \label{tg:basics:openHclosed}
            Ist $H\leq G$ eine offene Untergruppe, so ist $H$ auch
            abgeschlossen.
        \item \label{tg:basics:KKcompact}
            Sind $K_1,K_2\subset G$ kompakt, so auch $K_1K_2$.
    \end{enumerate}
\end{thLemma}

\begin{proof}
    Seien $U\in\frakU$ und $H\leq G$ beliebig.
    \begin{enumerate}[(a)]
        \item
            Da $e^{-1} = e$ gilt und Invertieren stetig ist, ex. nach der
            lokalen Charakterisierung von Stetigkeit eine Umgebung $T\in\frakU$,
            so dass $T^{-1} \subset U$ gilt.
            Dann leistest $V \defeq T \cap T^{-1}$ offenbar das Gewünschte.
        
        \item
            Es gilt $e\mkern2mu e=e$ und weil die Multiplikation stetig ist, ex. wieder
            eine Umgebung $T\subset G\times G$ von $(e,e)\in G\times G$, so dass 
            %$\bigl( G\times G \to G,\; (a,b)\mapsto ab \bigr)(T) \subset U$ gilt.
            für alle $(t_1,t_2)\in T$ das Produkt $t_1t_2$ in $U$ liegt.
            Nach Definition der Produkttopologie gibt es also
            $T_1,T_2\in\frakU$, so dass $T_1\times T_2 \subset T$ erfüllt ist.
            Setzte dann $V\defeq T_1\cap T_2$, dann gilt offenbar $VV\subset U$.

        \item
            Wähle zunächst eine Umgebung $V'\in\frakU$ nach
            \ref{tg:basics:VV}. Wähle dann nach \ref{tg:basics:symV} eine
            symmetrische Umgebung $V\in\frakU$ mit $V\subset V'$. Dann gilt
            natürlich wegen $V\subset V'$: $VV \subset V'V' \subset U$.

        \item
            Wir sehen zunächst ein, dass die Abbildung
            \begin{align*}
                \phi\colon G\times G &\to G,    \\
                            (a,b)    &\mapsto ab^{-1}
            \end{align*}
            als Komposition stetiger Abbildungen auch stetig ist. Wir verwenden
            nun das bekannte Kriterium, dass $\setclosure H$ eine Untergruppe
            von $G$ ist, wenn für alle $x,y\in\setclosure{H}$ auch
            $xy^{-1}\in\setclosure{H}$ gilt.
            
            Seien nun also $x,y \in \setclosure{H}$ beliebig und sei $U\subset G$
            eine beliebige Umgebung von $xy^{-1}$. Weil obiges $\phi$ stetig
            ist und offenbar $\phi(x,y)=xy^{-1}$ gilt, existiert eine Umgebung
            $T\subset G$ von $(x,y)$ derart, dass $\phi(T) \subset U$ gilt. Nach
            Definition der Produkttopologie gibt es dann Umgebungen $T_1$ bzw.
            $T_2$ von $x$ bzw. $y$, so dass $(x,y) \in T_1\times T_2 \subset T$
            erfüllt ist. Wegen $x,y\in\setclosure{H}$ erhalten wir: 
            $T_1\cap H\neq\emptyset$ und $T_2\cap H\neq\emptyset$. Es folgt
            $\phi(T_1\times T_2)=T_1 T_2^{-1}\cap H\neq\emptyset$ und damit auch
            $U\cap H\neq\emptyset$, womit $xy^{-1}$ im Abschluss von $H$ liegt. 

        \item
            Sei $H$ zusätzlich offen. Wir betrachten nun die Linksnebenklassen
            $gH$ für $g\in G$. Da $H$ offen ist, ist auch $gH$ offen für alle
            $g\in G$ und bekannterweise kann $G$ als disjunkte Vereinigung aller
            Linksnebenklassen geschrieben werden. Damit gilt:
            \[ \bigcup_{x\in G\setminus H} xH \]
            ist offen als Vereinigung offener Mengen und somit ist das
            Komplement in $G$, also gerade $H$, abgeschlossen.

        \item
            Für kompakte Teilmengen $K_1,K_2\subset G$ ist auch $K_1\times K_2$
            in $G\times G$ bezüglich der Produkttopologie kompakt und damit
            folgt die Kompaktheit von $K_1K_2$ daraus, dass dies das Bild von
            $K_1\times K_2$ unter der stetigen Gruppenmultiplikation ist.
    \end{enumerate}
\end{proof}

Meist wünscht man sich von der Topologie einer Gruppe $G$, dass diese
hausdorffsch ist. In vielen Anwendungen wird dies sowieso der Fall sein, aber
falls nicht, so gehen wir nun kurz darauf ein, wie wir diesen Mangel mit einer
recht einfachen Konstruktion beheben können. Zunächst stellen wir jedoch fest,
dass jede topologische Gruppe das Trennungsaxiom~$T_3$ erfüllt; dafür benutzen
wir das folgende Lemma:
\begin{thLemma}%
    [Charakterisierung von \texorpdfstring{$T_3$}{T3} durch Umgebungsbasen]
    \label{tg:t3basen}
    %
    Ist $X$ ein topologischer Raum, so erfüllt $X$ genau dann das
    Trennunsaxiom~$T_3$, wenn jeder Punkt aus $X$ eine Umgebungsbasis aus
    abgeschlossenen Mengen besitzt, d.\,h. wenn gilt:
    \\
    Für alle $x\in X$ und alle Umgebungen $U$ von $x$ existiert eine
    abgeschlossene Umgebung $V$ von $x$ mit $x\in V \subset U$.
\end{thLemma}
Der Beweis ist einfach und wird an dieser Stelle dem Leser zur Übung überlassen.
% TODO: Buchreferenz suchen und einfügen

\begin{thLemma}[Topologische Gruppen erfüllen \texorpdfstring{$T_3$}{T3}]
    Ist $G$ eine topologische Gruppe, so genügt die unterliegende Topologie von
    $G$ dem Trennungsaxiom~$T_3$.
\end{thLemma}

\begin{proof}
    Sei $U\in\frakU$, dann gibt es nach \mycref{tg:basics:symVV} ein
    symmetrisches $V\in\frakU$, so dass $VV\subset U$ gilt. Wir zeigen nun, dass
    dann auch $\setclosure{V}$ in $U$ enthalten ist. Sei also
    $x\in\setclosure{V}$. Dann gilt $x\in xV$ und wegen $x\in\setclosure{V}$ ist
    der Schnitt von $xV$ mit $V$ nicht leer, d.\,h es gibt $v_1,v_2\in V$ mit
    $xv_1 = v_2$. Daraus folgt aber sofort: $x = v_2v_1^{-1} \in VV^{-1} = VV
    \subset U$. Also gilt $e\in V \subset \setclosure{V} \subset U$.
    \\
    Wie wir schon zuvor angemerkt haben, überträgt sich diese Eigenschaft nun
    auf alle Punke $g\in G$, denn zu einer Umgebung $W$ von $g$ können wir
    $g^{-1}W\in\frakU$ betrachten und dann die abgeschlossene Umgebung durch
    Linksmultiplikation mit $g$ wieder zu einer Umgebung um $g$ machen. 
    \\
    Mit \cref{tg:t3basen} folgt nun die Behauptung.
    \\
\end{proof}

Da in topologischen Räumen, die $T_3$ erfüllen, die Trennungsaxiome $T_0$, $T_1$
und $T_2$ äquivalent sind (einfache Rechnung), erhalten wir somit:
\begin{thKorollar}%
    [Äquivalenz von \texorpdfstring{$T_0$, $T_1$ und $T_2$}{T0, T1 und T2}
     bei topologischen Gruppen]
    %
    Eine topologische Gruppe, die $T_0$ oder $T_1$ erfüllt, 
    ist schon hausdorffsch und damit insbesondere ein \emph{regulärer Raum}. 
    (Umgekehrt impliziert $T_2$ sowieso immer $T_1$ und $T_1$ immer $T_0$.)
\end{thKorollar}

Sollte eine topologische Gruppe~$G$ noch nicht hausdorffsch sein, so können wir
$G$ durch eine geeignete Quotientengruppe ersetzen und mit dieser
weiterarbeiten. Für eine Untergruppe $H$ von $G$ sei also $G/H$ die Menge der
Linksnebenklassen und $\pi\colon G \to G/H,\; g\mapsto gH$ die kanonische
Projektion. Dann statten wir $G/H$ mit der Quotiententopologie aus, d.\,h.
$U'\subset G/H$ ist genau dann offen, wenn $\pi^{-1}(U') \subset G$ offen in $G$
ist, wodurch wir wieder einen topologischen Raum erhalten. (Achtung: $G/H$
muss zunächst keine Gruppe bilden; dazu muss $H$ zusätzlich ein Normalteiler
sein!) Aus \cref{tg:translates} folgt, dass $\pi$ eine offene Abbildung ist,
denn: Sei $U \subset G$ offen, dann ist $\pi^{-1}\bigl( \pi(U) \bigr)
= \pi^{-1}\bigl( \{ uH \Mid u\in U \} \bigr) = UH$, was offen in $G$ ist.
Wir können nun Folgendes zeigen:
%
\begin{thLemma}[Topologische Gruppen und Quotientenräume]
    \label{tg:quot}
    %
    Sei $H$ eine Untergruppe der topologischen Gruppe $G$.
    %
    \begin{enumerate}[a)]
        \item\label{tg:quot:hausdorff}
            Ist $H$ abgeschlossen, so ist $G/H$ hausdorffsch.
        \item\label{tg:quot:lcompact}
            Ist $G$ lokalkompakt, so auch $G/H$.
        \item\label{tg:quot:topogrp}
            Ist $H$ ein Normalteiler von $G$, so ist $G/H$ wieder eine
            topologische Gruppe.
    \end{enumerate}
\end{thLemma}

\begin{proof}\hfill
    \begin{enumerate}[(a)]
        \item
            Sei also $H\leq G$ abgeschlossen. Seien $x,y\in G$ mit 
            $xH, yH\in G/H$, so dass $xH\neq yH$. Da $\pi$ offen ist,
            genügt es zu zeigen, dass
            wir Umgebungen $U,V$ von $x$ bzw. $y$ in $G$ finden können, so dass
            $\pi(U)\cap\pi(V)=\emptyset$ gilt. Wegen $xH\neq yH$ gilt auch
            $x\notin yH$ und somit $e\notin yHx^{-1}$. Weil $H$ abgeschlossen
            ist, ist auch $yHx^{-1}$ abgeschlossen und es gilt 
            $e\in G\setminus (yHx^{-1})$, wobei letztere Menge offen ist. Nach
            \mycref{tg:basics:symVV} gibt es somit ein symmetrisches
            $T\in\frakU$ mit $TT \subset G\setminus (yHx^{-1})$, also
            insbesondere $TT \cap yHx^{-1} = \emptyset$ bzw.
            $H \cap (y^{-1} TT x) = \emptyset$~~$(\ast)$.
            
            Wir setzen nun $U \defeq Tx$ und $V\defeq Ty$ und behaupten, dass
            diese Umgebungen obige Forderung erfüllen. Angenommen der Schnitt
            von $\pi(Tx)$ und $\pi(Ty)$ ist nicht leer. Dann gibt es 
            $t_1,t_2\in T$ und $h_1,h_2\in H$, so dass $t_1xh_1 = t_2yh_2$ gilt.
            Nach Umformung erhalten wir: $h_2h_1^{-1} = y^{-1}t_2^{-1} t_1 x$.
            Da $H$ eine Untergruppe ist, gilt natürlich $h_2h_1^{-1} \in H$, und
            da $T$ symmetrisch ist, ist auch
            $y^{-1}t_2^{-1} t_1 x \in y^{-1}TTx$ wahr. Dies is aber ein
            Widerspruch zu $(\ast)$. Also muss doch $\pi(U)\cap\pi(V)=\emptyset$
            gelten, was die Behauptung zeigt.
            
        \item
            Sei $xH\in G/H$ und sei $K\in\frakU$ eine kompakte Umgebung von $e$
            in $G$. Dann ist auch $xK$ und wegen der Stetigkeit von $\pi$ auch
            $\pi(xK)$ kompakt und letzteres ist eine Umgebung von $\pi(x)=xH$ in
            $G/H$.
            
        \item
            Wegen $H\triangleleft G$ ist $G/H$ wieder eine Gruppe, welche wir
            wie zuvor mit der Quotiententopologie als topologischen Raum
            betrachten. Es bleibt zu zeigen, dass die Gruppenoperationen auf
            $G/H$ stetig sind.
            
            Seien dazu $xH,yH\in G/H$ und sei $U'\subset G/H$ eine Umgebung von
            $xyH \in G/H$. Weil $\pi$ stetig ist, gibt es dann eine Umgebung
            $U\subset G$ von $xy\in G$, so dass $\pi(U)$ Teilmenge von $U'$ ist.
            Nach Definiton der Produkttopologie und der Stetigkeit der
            Multiplikation in $G$ existieren nun Umgebungen $V_x$ bzw. $V_y$ von
            $x$ bzw. $y$, so dass $V_xV_y \subset U$ gilt. Dann sind $\pi(V_x),
            \pi(V_y)$ Umgebungen von $xH$ bzw. $yH$ derart, dass
            $\pi(V_x)\,\pi(V_y)\subset U'$ erfüllt ist (womit die Multiplikation
            in $G/H$ stetig ist), denn es gilt:
            \[ \pi(V_x)\,\pi(V_y) = \{ v_xH \Mid v_x\in V_x \} \, \{ v_yH \Mid
                v_y\in V_y \} = \{ v_xv_yH \Mid (v_x,v_y)\in V_x\times V_y \}
            \]
            und nach Konstruktion gilt $\pi(V_xV_y) \subset \pi(U) \subset U'$.
            
            Sei erneut $xH\in G/H$ und nun $U'\subset G/H$ eine Umgebung
            von $xH$. Dann gibt es wegen der Stetigkeit von $\pi$ wieder eine
            Umgebung $U\subset G$ von $x$, so dass $\pi(U)$ in $U'$ enthalten
            ist. Wegen der Stetigkeit von Invertieren in $G$ gibt es eine
            Umgebung $V\subset G$ von $x^{-1}$, so dass $V^{-1} \subset U$.
            Dann gilt: $\pi(V)$ ist eine Umgebung von $x^{-1}H$ in $G/H$ und
            $\pi(V)^{-1} \subset U'$, denn:
            \[ \pi(V)^{-1} = \{ vH \Mid V \}^{-1} = \{ \tilde{v}H \Mid
                \tilde{v}\in V^{-1} \} = \pi(V^{-1}) \subset \pi(U) \subset U' 
            \]
    \end{enumerate}
\end{proof}

Wir erhalten daraus:
\begin{thKorollar}%
    [\enquote{Hausdorffizierung} topologischer Gruppen]
    \label{tg:hausdorffizierung}
    %
    Ist $G$ eine (lokalkompakte) topologische Gruppe, welche \emph{nicht}
    hausdorffsch ist, so erhalten wir mittels $G/\setclosure{\{e\}}$ eine
    hausdorffsche (lokalkompakte) topologische Gruppe.
\end{thKorollar}

\begin{proof}
    Sei $H\defeq \setclosure{\{e\}}$. 
    Zunächst ist $H$ wegen \mycref{tg:basics:closureH} eine Untergruppe von $G$.
    Nach \mycref{tg:quot:hausdorff} ist somit $G/H$ hausdorffsch, da $H$
    abgeschlossen ist, und nach \mycref{tg:quot:lcompact} ist $G/H$ auch
    lokalkompakt, falls $G$ lokalkompakt ist. Es bleibt zu zeigen, dass $H$ ein
    Normalteiler von $G$ ist.
    \\
    Angenommen $H$ ist kein Normalteiler von $G$. Dann gibt es ein $g\in G$,
    so dass $gHg^{-1}$ keine Teilmenge von $H$ ist. Weil $H$ abgeschlossen ist,
    ist auch $gHg^{-1}$ abgeschlossen und wir erhalten somit eine neue
    Untergruppe $H \cap (gHg^{-1})$ von $G$, welche auch abgeschlossen ist.
    Diese wäre dann eine echte Teilmenge von $H$, was im Widerspruch zur
    der Tatsache steht, dass $H$ offensichtlich die (bezüglich Inklusion)
    kleinste abgeschlossene Untergruppe von $G$ ist. Also war die Annahme falsch
    und es muss schon $H\triangleleft G$ gelten.
    \\
\end{proof}

Diese Folgerung erlaubt es uns also, in Zukunft immer davon auszugehen, dass die
unterliegende Topologie von $G$ hausdorffsch ist, denn sonst bilden wir einfach
$G/\setclosure{\{e\}}$ und arbeiten damit weiter. (Betrachtet man Borel-messbare
Funktionen auf $G$, so kann dieser Schritt insbesondere dadurch gerechtfertigt
werden, dass man Folgendes zeigen kann: Ist $f$ eine Borel-messbare Funktion auf
$G$, so ist $f$ konstant auf den (Links-)Nebenklassen von $\setclosure{\{e\}}$.
D.\,h. effektiv ist $f$ dann sowieso eine Funktion auf $G/H$.)
% ^ TODO: Buchreferenz suchen und hinzufügen


\section{(Stetige) Funktionen}
In diesem Abschnitt werfen wir einen kurzen Blick auf die Welt der (stetigen)
Funktionen auf einer topologischen Gruppe. Die Begriffe, die wir gleich
definieren werden, spielen unter anderem eine bedeutende Rolle beim Beweis der
Existenz und Eindeutigkeit des Haar-Maßes (siehe später). % TODO: verlinken
Auch in diesem Abschnitt sei $G$ eine beliebige topologische Gruppe.

\begin{thNotation}[Funktionenräume, Träger]
    \label{tg:notation:contcompsupp}
    %
    Es bezeichne $\continuous(G)$ den Raum der komplexwertigen, stetigen
    Funktionen auf $G$. Ist $f\in \continuous(G)$, so bezeichnen wir mit 
    $\supp(f) \defeq \setclosure{ \{ x\in G \Mid f(x) \neq 0 \} }$
    den Träger von $f$ und mit $\contcomp(G)$ denjenigen Teilraum von
    $\continuous(G)$, der alle stetigen Funktionen mit kompaktem Träger umfasst.
    Weiter setzen wir
    \[ \contcompplus(G) \defeq \{ f\in \contcomp(G) \Mid 
        f \geq 0 \;\wedge\; \supnorm{f} > 0 \}
    . \]
    (In $f\geq 0$ fordern wir implizit, dass $f$ reellwertig ist.)
    Besteht keine Verwechslungsgefahr, so schreiben wir außerdem kurz
    $\contcompplus$ für $\contcompplus(G)$.
\end{thNotation}

\begin{thDef}[Translation von Funktionen]
    Ist $f$ eine Funktion auf $G$ und $y\in G$, so definieren wir die 
    \emph{Linkstranslation von $f$ mit $y$} durch
    \[ L_y f \defeq \bigl( x\mapsto f(y^{-1}x) \bigr) \]
    und die \emph{Rechtstranslation von $f$ mit $y$} durch
    \[ R_y f \defeq \bigl( x\mapsto f(xy) \bigr) . \]
\end{thDef}
%
Der Sinn der Invertierung von $y$ bei $L_y$ ergibt sich, sobald man Verkettungen
von Translationen betrachtet. Nun gilt nämlich für $y,z\in G$:
\[  L_y L_z f = L_{yz} f  \qqundqq  R_y R_z f = R_{yz} f  , \]
d.\,h. die Abbildungen $G\ni y \mapsto L_y$ bzw. $G\ni y\mapsto R_y$ sind
Gruppenhomomorphismen von $G$ in die Gruppe der Links- bzw. Rechtstranslationen
auf $G$.

\emph{Achtung:} In der Literatur wird dies nicht einheitlich eingeführt. Bei manchen
Autoren wird $y$ bei Linkstranslationen \emph{nicht} invertiert, was aber zur
Folge hat, dass die Formeln aus dem vorherigen Absatz verändert werden müssen.

\begin{thDef}[Links-/Rechts-gleichmäßig stetig]
    Sei $f$ eine komplexwertige Funktion auf $G$. 
    Wir nennen $f$ dann \emph{links-gleichmäßig stetig} (bzw.
    \emph{rechts-gleichmäßig stetig}), wenn es für alle $\epsilon\in\R[>0]$ eine
    Umgebung $U\in\frakU$ von $e$ gibt, so dass für alle $y\in U$ die
    Ungleichung $\supnorm{f-L_y f} < \epsilon$ 
    (bzw. $\supnorm{f-R_yf} < \epsilon$) erfüllt ist.
\end{thDef}

\emph{Achtung:} Auch hier verfahren die Autoren unterschiedlich. Manchmal werden beide
Begriffe genau vertauscht eingeführt.

% TODO: ggf. Hinweis und Literaturverweis auf 'uniforme Strukturen' einfügen

\begin{thLemma}%
    [Gleichmäßige Stetigkeit von stetigen Funktionen mit kompaktem Träger]
    \label{tg:unicont}
    %
    Ist $f\in\contcomp(G)$, so ist $f$ sowohl links- als auch rechts-gleichmäßig
    stetig in obigem Sinne.
\end{thLemma}

\begin{proof}
    Sei $f\in\contcomp(G)$ und $\epsilon\in\R[>0]$.
    Da beide Begriffe analog definiert sind, genügt es, eine der Behauptungen zu
    beweisen; die andere verläuft dann absolut analog. Wir zeigen also nur, dass
    $f$ rechts-gleichmäßig stetig ist.
    \\
    Sei $K\defeq\supp(f)$. Zu jedem $x\in K$ betrachte dann die Abbildung
    \[ G\ni y \mapsto \abs{ f(x)-f(xy) } .\]
    Diese ist stetig als Komposition stetiger Abbildungen, und $e$ wird offenbar
    auf $0$ abgebildet. Daher existiert ein $U_x\in\frakU$, so dass
    \[  \forAll_{y\in U_x}\qquad
        \abs{f(x)-f(xy)} < \frac{\epsilon}{2}
        \tag{$\ast$}\label{tg:unicont:p:haveineq}
    \]    
    erfüllt ist. Zu jedem $U_x$ finden wir dann nach \mycref{tg:basics:symVV}
    ein symmetrisches $V_x\in\frakU$ mit $V_x \subset V_x V_x \subset U_x$.
    Offenbar gilt dann:
    \[  K \subset \bigcup_{x\in K} xV_x \qqtextqq{und somit schon}
        K \subset \bigcup_{i=1}^n  x_iV_i 
    \]
    für endlich viele $x_i$, da $K$ nach Voraussetzung kompakt ist.
    Setze nun
    \[ U \defeq \bigcap_{i=1}^n V_i \,, \]
    dann ist $U\in\frakU$ und wir behaupten, dass $U$ die gesuchte Umgebung ist,
    d.\,h., dass für alle $y\in U$ und alle $x\in G$ gilt: 
    \[ \abs{ f(x) - f(xy) } < \epsilon  
        \tag{$\star$}\label{tg:unicont:p:wantineq}
    \]
    Um dies zu zeigen, unterscheiden wir mehrere Fälle. Zunächst gilt
    \eqref{tg:unicont:p:wantineq} offensichtlich, wenn weder $x$, noch $xy$ in $K$
    liegen. Sei nun also $x\in K$. Dann gibt es ein $j\in\{1,\ldots,n\}$, so
    dass $x\in x_jV_j$ und damit $x_j^{-1}x \in V_j$ gilt. Außerdem erhalten
    wir $xy = x_j(x_j^{-1}x)y \in x_jV_jV_j \subset x_jU_{x_j}$, denn $y$ ist
    nach Definition von $U$ auch in $V_j$. Es folgt mit der Dreiecksungleichung:
    \[ \abs{ f(x) - f(xy) } 
        \leq \underbrace{ \abs{f(x_j) - f(xy)} }_{
                \overset{\eqref{tg:unicont:p:haveineq}}{<} \frac{\epsilon}{2},
                \text{ wegen } xy\in x_jU_{x_j} 
             }
          +  \underbrace{ \abs{f(x) - f(x_j)} }_{
                \overset{\eqref{tg:unicont:p:haveineq}}{<} \frac{\epsilon}{2},
                \text{ wähle } y=x_j^{-1}x \text{ in \eqref{tg:unicont:p:haveineq}}  
             }
        < \epsilon
    \]
    Also ist auch in diesem Fall \eqref{tg:unicont:p:wantineq} erfüllt.
    Sei nun $x\notin K$, d.\,h. $f(x)=0$, und $xy\in K$, so dass wir uns noch um
    $f(xy)$ kümmern müssen. Wieder muss es ein $j\in\{1,\ldots,n\}$ geben, so
    dass $xy\in x_jV_j\subset x_jU_{x_j}$ und damit $x_j^{-1}xy\in V_j$ sowie 
    $x_j^{-1}x\in V_jy^{-1} \subset V_j V_j \subset U_{x_j}$ gilt. 
    Wir erhalten dann durch eine ähnliche Abschätzung die Gültigkeit von
    \eqref{tg:unicont:p:wantineq}:
    \[ \abs{ f(x) - f(xy) } = \abs{f(xy)}
        \leq \underbrace{ \abs{f(x_j) - f(xy)} }_{
                \overset{\eqref{tg:unicont:p:haveineq}}{<} \frac{\epsilon}{2},
                \text{ wegen } xy\in x_jU_{x_j} 
             }
          +  \underbrace{ \abs{f(x_j)} }_{
                \overset{\eqref{tg:unicont:p:haveineq}}{<} \frac{\epsilon}{2},
                \text{ wähle } y=x_j^{-1}x \text{ in \eqref{tg:unicont:p:haveineq}}  
             }
        < \epsilon
    \]
\end{proof}

(Anstatt sich darauf zu berufen, dass der Beweis analog abläuft, kann man auch
mit einem Trick argumentieren, den man bei
Elstrod\cite[Kap.\,VIII,\;\S3,\;3.8]{bookc:elstrod11}
findet: Man lasse die Topologie auf $G$ unverändert, aber betrachte die
umgekehrte Multiplikation: $x\bullet y \defeq yx$. Man überlegt sich dann, dass
man dadurch die für \enquote{rechts"~} bewiesenen Aussagen auf \enquote{links"~}
übetragen kann und umgekehrt~\ldots)

Für spätere Zwecke halten wir noch folgende Aussage fest:
\begin{thLemma}%
    [\texorpdfstring{$\contcomp$}{Cc}-Abbildungen unter Translationen]
    \label{tg:Cctranslat}
    %
    Ist $f\in\contcomp(G)$, so ist für alle $y\in G$ auch
    $L_yf\in\contcomp(G)$ und $R_yf\in\contcomp(G)$. 
\end{thLemma}

\begin{proof}
    Es seien $y\in G$,
    $\phi(x) \defeq y^{-1}x$ und $\psi(x) \defeq xy$.
    Dann gilt
    \[  L_y f = f \circ \phi 
        \qqundqq
        R_y f = f \circ \psi
    . \]
    Nach \cref{tg:homoeom} sind $\phi$ und $\psi$ Homöomorphismen und daher
    folgt die Behauptung aus der (einfach nachzurechnenden) Tatsache, dass dann
    $\supp(f\circ\phi) = \phi^{-1}\bigl(\supp(f)\bigr)$ für $\phi$ und analog 
    für $\psi$ gilt, wobei Kompakta wegen der Stetigkeit der Umkehrabbildung wieder auf
    Kompakta abgebildet werden.
    \\
\end{proof}


\section{Invariante Maße und Linearformen}
Von nun an beschäftigen wir uns hauptsächlich mit LKH-Gruppen. Für den Rest des
Abschnitts sei $G$ eine solche Gruppe.

\begin{thDef}[Links-/Rechtsinvariante(s) Maß / Linearform]\hfill\\
    Wir nennen ein Maß $\mu\colon \borelsigmaalg(G) \to [0,\infty]$
    \emph{linksinvariant} (bzw. \emph{rechtsinvariant}), 
    wenn für alle $x\in G$ und alle $B\in\borelsigmaalg(G)$ gilt:
    \[  \mu(xB) = \mu  
        \qquad\qquad\bigl(\text{bzw. } 
        \mu(Bx) = \mu(B) 
        \,\bigr)
    \]
    Weiter nennen wir ein lineares Funktional $I$ auf $\contcomp(G)$ 
    \emph{linksinvariant} (bzw. \emph{rechtsinvariant}),
    wenn für alle $y\in G$ und alle $f\in\contcomp(G)$ gilt:
    \[  I(L_yf) = I(f) 
        \qquad\qquad\bigl(\text{bzw. } 
        I(R_yf) = I(f)
        \,\bigr)
    \]
\end{thDef}

\medskip
Der Vollständigkeit halber wiederholen wir (ohne Beweis) 
ein paar Definition und Sätze, die wir im Folgenden und später im Existenz- und
Eindeutigkeitsbeweis des Haar-Maßes  % TODO: verlinken
benötigen werden.

\begin{thDef}[Radon-Maß, lokal-endlich, von innen regulär]
    \label{tg:def:radonmeasure}
    Ein Radon-Maß $\mu$ auf einem topologischen Raum $X$ ist ein
    \emph{lokal-endliches} Maß $\mu\colon\borelsigmaalg(X)\to[0,\infty]$, das
    zusätzlich \emph{von innen regulär} ist.
    Dabei bedeutet:
    \begin{description}[align=left, 
                        itemindent=15pt, 
                        leftmargin=0pt, 
                        itemsep=0pt,
                        topsep=0.3\baselineskip]
        \item[lokal-endlich:]
            Für alle $y\in X$ ex. eine (offene) Umgebung $U\subset X$ von
            $y$ mit $\mu(U) < \infty$.
            %Für alle $x\in X$ gibt es eine (offene) Umgebung $U\subset X$ von
            %$x$, so dass $\mu(U)$ endlich ist.
        \item[von innen regulär:]
            Für alle $B\in\borelsigmaalg(X)$ gilt:
            $\mu(B) = \sup\{ \mu(K) \Mid K\subset B,\; K\text{ kompakt} \}$.
    \end{description}
\end{thDef}

\begin{thSatz}[Allgemeine Transformationsformel]
    \label{tg:trafo}
    %
    Seien $(X,\mathcal{A},\mu)$ ein Maßraum, $(Y,\mathcal{B})$ ein
    messbarer Raum, $\phi\colon X\to Y$ eine $\mathcal{A}$-$\mathcal{B}$-messbare
    Abbildung und $f\colon Y\to\C$ eine $\mathcal{B}$-messbare Funktion.
    Ist dann $\mu_\phi \defeq \mu \circ \phi^{-1}$ das von $\mu$ und $\phi$
    induzierte Bildmaß auf $(Y,\mathcal{B})$, dann gilt:
    \\
    Es ist $f$ genau dann bzüglich $\mu_\phi$ integrierbar, wenn $f\circ\phi$
    bezüglich $\mu$ integrierbar ist und für die Integrale gilt:
    \[ \int_Y f \dif{\mu_\phi} = \int_X f\circ\phi \dif{\mu} \]
\end{thSatz}

\begin{thSatz}%
    [\enquote{\texorpdfstring{Kleiner $\contcomp$}{Cc}-Fubini} 
        für lokalkompakte Hausdorffräume]
    \label{tg:Ccfubini}
    %
    Seien $X,Y$ lokalkompakte Hausdorffräume und seien $\mu$ bzw. $\nu$
    Radon-Maße auf $X$ bzw. $Y$. Dann gilt für alle $f\in\contcomp(X\times Y)$:
    \[   \int_X \int_Y f(x,y) \dif{\nu(y)} \dif{\mu(x)}
        =\int_Y \int_X f(x,y) \dif{\mu(x)} \dif{\nu(y)}
    \]
    Oder kurz auch: $\iint f \dif\nu\dif\mu = \iint f \dif\mu\dif\nu$.
\end{thSatz}

\begin{thSatz}[Riesz'scher Darstellungssatz]
    \label{tg:riesz}
    %
    Ist $X$ ein lokalkompakter Hausdorffraum und $I$ ein positives lineares
    Funktional auf
    $\contcomp(X)$\footnote{analog zu \cref{tg:notation:contcompsupp}}, so gibt es genau
    ein Radon-Maß $\mu$ auf $X$ mit
    \[ I(f) = \int_X f \dif\mu 
        \qquad\text{für alle $f\in\contcomp(X)$.}
    \]
    Dabei bedeutet \emph{positives Funktional}, dass für alle
    $f\in\contcomp(X)$ mit $f\geq0$ auch $I(f)\geq0$ gilt.
\end{thSatz}

Die Definition(en) (in möglicherweise leicht anderer Form)
sowie die allgemeine Transformationsformel und den
Darstellungssatz von Riesz findet man in den meisten Büchern zur Maß- und
Integrationstheorie. Für die (auf das Nötige) beschränkte Variante des Satzes
von Fubini für lokalkompakte Räume, kann man bei Folland\cite{bookc:folland99}
einen Beweis finden. Konkret sei auf folgende Referenzen verwiesen:
\begin{itemize}[itemsep=0pt,
                topsep=0.3\baselineskip]
    \item  \cref{tg:def:radonmeasure}: 
        Elstrod\cite[Kap.\,VIII,\;\S1]{bookc:elstrod11}
    \item  \cref{tg:trafo}:
        Elstrod\cite[Kap.\,V,\;\S3,\;3.1]{bookc:elstrod11}
        oder
        Bogachev\cite[Chapter\,3,\;3.6]{bookc:bogachev07}
    \item  \cref{tg:Ccfubini}:
        Folland\cite[\S7,\;7.22]{bookc:folland99}
        % TODO: ggf. weitere sinnvolle Referenzen angeben
    \item  \cref{tg:riesz}:
        Elstrod\cite[Kap.\,VIII,\;\S2,\;2.5]{bookc:elstrod11}
        oder
        Folland\cite[\S7,\;7.2]{bookc:folland99}
        % TODO: vorherigen Vortrag als Referenz hinzufügen
\end{itemize}

\bigskip
Dank des Darstellungssatzes von Riesz bekommen wir sofort auch den Zusammenhang
zwischen invarianten Maßen und Linearformen auf LKH-Gruppen:
\begin{thLemma}[Invariante Radon-Maße entsprechen invarianten Linearformen]
    Ist $\mu$ ein linksinvariantes Radon-Maß auf $G$, so existiert genau ein
    linksinvariantes positives lineares Funktional $I$ auf $\contcomp(G)$, 
    so dass für alle $f\in\contcomp(G)$ gilt:
    \[ I(f) = \int_G f \dif\mu \]
    Umgekehrt erhält man aus dieser Formel auch für jedes linksinvariante
    Radon-Maß ein entsprechendes linksinvariantes positives lineares Funktional.
\end{thLemma}

\begin{proof}
    \ldots % TODO
\end{proof}


\section{Das Haar'sche Maß}
\begin{thDef}[Haar-Maß]
    Ist $G$ eine LKH-Gruppe, so ist ein \emph{linkes Haar-Maß} (bzw.
    \emph{rechtes Haar-Maß}) auf $G$ ein linksinvariantes (bzw.
    rechtsinvariantes) Radon-Maß auf $G$, welches von null verschieden ist.
\end{thDef}





































\chapter{Existenz und Eindeutigkeit des Haar'schen Maßes}
Wir wollen nun die Existenz und Eindeutigkeit (bis auf einen positiven Faktor)
eines Haar-Maßes auf LKH-Gruppen zeigen. 
% (Im Folgenden meinen wir mit \enquote{Eindeutigkeit des Haar-Maßes} 
% immer die Eindeutigkeit bis auf einen konstanten Faktor.)
Für den Existenzbeweis halten wir uns an den erstmals 1940 von 
\emph{Andr\'e Weil} in allgemeiner Form veröffentlichten Beweis für beliebige
LKH-Gruppen, wie man ihn etwa bei
Elstrod\cite[Kap.\,VIII,\;\S3,\;3.12]{bookc:elstrod11} 
oder auch bei Folland\cite[\S11,\;11.8]{bookc:folland99} findet.
Dieser benutzt das \emph{Auswahlaxiom} in Form des
\emph{Satzes von Tychonoff}, weshalb es an dieser Stelle erwähnenswert ist, dass
\emph{Henri Cartan} (ebenfalls 1940) auch einen Beweis gefunden hat, der ohne
selbiges auskommt. (Siehe dazu auch: Alfsen\cite{artcle:alfsen63}.)
Für den Eindeutigkeitsbeweis verwenden wir ein Argument, das hauptsächlich auf
dem Gebrauch des \emph{Satzes von Fubini} beruht, siehe beispielsweise
Folland\cite[\S11,\;11.9]{bookc:folland99}. Auch hier gibt es alternative
Beweise, die mit elementareren Mitteln auskommen. Der ursprüngliche Beweis von
\emph{A. Weil} macht zum Beispiel keinen Gebrauch vom Satz von Fubini und ein
Beweis nach diesem Schema findet man im oben schon zitierten Satz bei Elstrod.

\medskip
Da wir schon gezeigt haben, dass sich Haar-Maße und Haar-Integrale entsprechen,
genügt es, die Existenz und (im Wesentlichen) Eindeutigkeit eines Haar-Integrals
zu zeigen, und dies ist auch der übliche Ansatz. Im Übrigen meinen wir hier und
im Rest dieses Kapitels stets \emph{linkes} Haar-Maß und \emph{linkes}
Haar-Integral, wenn wir es nicht explizit spezifizieren. Um nun eine Motivation
für die weiter unten auftrende Konstruktion zu geben, betrachten wir folgendes
Szenario: Sei $G$ eine LKH-Gruppe und seien $K\subset G$ eine kompakte und
$U\in\frakU$ eine offene Menge. Weil wir ein von innen reguläres Maß suchen,
genügt es prinzipiell, das Maß auf kompakten Mengen zu kennen. Wir möchten nun
also ein \enquote{Maß} für $K$ angeben, was wir zum Beispiel unter Verwendung
von $U$ wie folgt tun können:
Weil $K$ kompakt ist, finden wir ein $n\in\N$ und endlich viele $x_i\in G$, so
dass $(x_iU)_{i\in\{1,\ldots,n\}}$ eine Überdeckung von~$K$ bildet. Definiert
man nun $(K : U)$ als das minimale derartige $n\in\N$, so misst $(K : U)$ in
gewissem Rahmen wie groß $K$ in Relation zu $U$ ist. Die Idee ist nun, $U$ immer
kleiner zu wählen, so dass $U$ auf $\{e\}$ zusammenschrumpft und man somit die
Menge $K$ mit immer weniger \enquote{Überlappungen} überdecken kann. Mittels
einer Normierung, also $(K : U)/(K_0 : U)$ für festes $K_0\subset G$, können wir
erhoffen, dass sich bei dem eben skizzierten Grenzprozess ein Wert für diesen
Quotienten einstellt und wir dadurch ein (offenbar linksinvariantes) Maß auf~$G$
bekommen.

Wie allerdings gerade schon angesprochen, ist es einfacher, die Sichtweise auf
Funktionale zu verlagern und in ähnlicher Herangehensweise ein Haar-Integral auf
einer beliebigen (aber festen) LKH-Gruppe zu konstruieren. Im Detail
funktioniert dies so:


\section{Existenz eines Haar-Maßes}
Sei ab jetzt $G$ eine fixierte LKH-Gruppe. Sind dann $f,\phi\in\contcompplus$,
so ist 
\[ V \defeq \bigl\{ x \in G \Mid \phi(x) > \thalf\supnorm\phi \bigr\} 
      = \phi^{-1}\Bigl( \bigl( \thalf\supnorm\phi, \mkern2mu\infty\bigr) \Bigr)
\]
eine offene Menge und weil $\supp(f)$ kompakt ist, gibt es ein $n\in\N$ und 
endlich viele $x_i\in G$, so dass $\supp(f)$ von $(x_iV)_{i\in\{1,\ldots,n\}}$ 
überdeckt wird. Dann gilt für $x\in\supp(f)$:
\[
    \biggl( \frac{2}{\supnorm\phi} \sum_{i=1}^n L_{x_i}\phi \biggr)(x)
    = \frac{2}{\supnorm\phi} \sum_{i=1}^n \phi(x_i^{-1}x)
\]
Wegen der Wahl von $x$ gibt es ein $x_k$ (mit $k\in\N$), so
dass $x\in x_kV$ bzw. $x_k^{-1}x\in V$ erfüllt ist. Daraus folgt:
\[  
    \frac{2}{\supnorm\phi} \sum_{i=1}^n \phi(x_i^{-1}x)
    \geq \frac{2}{\supnorm\phi} \, \phi(x_k^{-1}x)
    > \frac{2}{\supnorm\phi} \, \frac{\supnorm\phi}{2} 
    = 1 \geq \frac{f(x)}{\supnorm f}
\]
Trivialerweise ist die Ungleichung auch für $x\notin\supp(f)$ wahr, so dass
gilt:
\[  f \leq \frac{2\supnorm f}{\supnorm\phi} \sum_{i=1}^n L_{x_i}\phi  \]
Diese Idee nutzen wir nun zu einer allgemeineren Definition, bei der wir
unterschiedliche Koeffizienten für jeden Summanden zulassen:

\begin{thDef}[(Haar-)Überdeckungszahl]
    \label{pf:def:covernum}
    %
    Für $f,\phi \in\contcompplus$ definiert
    \[
        (f : \phi) \defeq 
        \inf \biggl\{\, \sum_{i=1}^n c_i  \cMid[\;]{\bigg} 
                     f \leq \sum_{i=1}^n c_i \, L_{x_i}\phi 
                     \qtextq{mit} n\in\N, \;\;
                     \forall i\in\{1,\ldots,n\}\colon 
                     c_i\in\R[>0] \wedge x_i\in G
            \biggr\}
    \]
    die sogenannte \emph{(Haar-)Überdeckungszahl} von $f$ und $\phi$.
\end{thDef}

Nach der Vorüberlegung ist die rechte Menge nicht leer und offenbar ist sie
durch~$0$ von unten beschränkt; somit ist die Überdeckungszahl wohldefiniert.
Weiter gelten folgende mehr oder weniger offensichtliche Aussagen:
%
\begin{thLemma}[Eigenschaften der Überdeckungszahl]
    \label{pf:covernumprops}
    %
    Für $f,g,\phi \in\contcompplus$ und $x\in G$ sowie $\lambda\in\R[>0]$ gilt:
    \begin{enumerate}[a)]\vspace{-4pt}
        \item
            Abschätzung nach unten:\quad
            $0 < \supnorm{f}/\supnorm{\phi} \leq (f : \phi)$
        \item
            Linksinvarianz im ersten Argument:\quad
            $(L_xf : \phi) = (f : \phi)$.
        \item
            Verträglichkeit mit Skalarmultiplikation:\quad
            $(\lambda f : \phi) = \lambda (f : \phi)$.
        \item
            Subadditvität:\quad
            $(f\mkern1mu{+}\mkern2mu g : \phi) \leq (f : \phi) + (g : \phi)$.
        \item\label{pf:covernumprops:reducing}
            \enquote{Abgeschätztes Kürzen}:\quad
            $(f : g)\,(g : \phi) \geq (f : \phi)$.
    \end{enumerate}
\end{thLemma}

\begin{proof}
    Seien alle Variablen wie in der Behauptung gewählt und seien weiter
    \[  f \leq \sum_{i=1}^n c_i \, L_{x_i}\phi
        \qqundqq
        g \leq \sum_{j=1}^m d_j \, L_{y_j}\phi
    \]
    (mit $c_i,d_j\in\R[>0],\; x_i,y_j\in G$ für alle $i,j$) 
    Ausdrücke wie in der Definition der Überdeckungszahl \pcref{pf:def:covernum}.
    \begin{enumerate}[(a)]
        \item 
            Bilden wir auf beiden Seiten der Ungleichung für $f$ die
            Supremumsnorm, so erhalten wir mit der Dreiecksungleichung:
            \[ \supnorm{f} \leq \sum_{i=1}^n c_i \,
                \underbrace{\supnorm{L_{x_i}\phi}}_{=\supnorm\phi}
            \]
            Durch Umstellen und Infimumsbildung folgt die Behauptung.
            
        \item
            Es gilt
            \[  f \leq \sum_{i=1}^n c_i \, L_{x_i}\phi 
                \qiffq
                L_xf \leq L_x \biggl( \sum_{i=1}^n c_i \, L_{x_i}\phi \biggr)
                        = \sum_{i=1}^n c_i \, L_{xx_i}\phi
            , \]
            woraus sofort die Behauptung folgt.
            
        \item
            Hier gilt analog:
            \[  f \leq \sum_{i=1}^n c_i \, L_{x_i}\phi 
                \qiffq
                \lambda f \leq \lambda \sum_{i=1}^n c_i \, L_{x_i}\phi
            \]
            
        \item
            Addition der Ungleichungen für $f$ und $g$ liefert:
            \[ f+g \leq 
                \sum_{i=1}^n c_i\,L_{x_i}\phi + \sum_{j=1}^m d_j\,L_{y_j}\phi
            \]
            Alle rechten Kombinationen von $c_i,x_i,d_j,y_j$ können offenbar
            auch mit der Wahl entsprechender $\gamma_k\in\R[\geq0],\; z_k\in G$
            in folgender Formel erreicht werden:
            \[ f+g \leq \sum_{k=1}^N \gamma_k \, L_{z_k}\phi \]
            Dies sind aber genau die Ausdrücke aus \cref{pf:def:covernum},
            welche wir zur Bestimmtung von $(f\mkern1mu{+}\mkern2mu g : \phi)$
            beachten müssen. Daraus folgt zunächst
            \[ (f\mkern1mu{+}\mkern2mu g : \phi) \leq 
                    \sum_{i=1}^n c_i + \sum_{j=1}^m d_j
            \]
            und durch Infimumsbildung wieder die Behauptung.
            
        \item
            Wir betrachten die Ungleichung für $f$, aber mit $g$ statt $\phi$, und
            benutzen dann die Ungleichung für $g$ und $\phi$, um folgendermaßen
            abzuschätzen:
            \[ f \leq \sum_{i=1}^n c_i \, L_{x_i}g 
                \leq \sum_{i=1}^n c_i \, L_{x_i}\biggl( 
                        \sum_{j=1}^m d_j \, L_{y_j}\phi
                    \biggr)
                = \sum_{i=1}^n\sum_{j=1}^m c_i d_j \, L_{x_iy_j}\phi
            \]
            Wegen
            \[ \sum_{i=1}^n\sum_{j=1}^m c_i d_j 
                =   \biggl( \sum_{i=1}^n c_i \biggr) \, 
                    \biggl( \sum_{j=1}^m d_j \biggr)
            \]
            folgt die Behauptung erneut durch Infimumsbildung.
    \end{enumerate}
\end{proof}

\medskip
Nun wählen wir ein $f_0\in\contcompplus$ beliebig, aber fest für den Rest des
Abschnitts. Dieses $f_0$ wird (wie $K_0$ in der Einleitung) die Rolle der
Normierung übernehmen, wie wir später sehen werden.

\begin{thDef}[Approximierende Funktionale]
    Zu jedem $\phi \in\contcompplus$ definieren wir ein Funktional $I_\phi$ wie folgt:
    \begin{align*}
        I_\phi\colon \contcompplus &\to         \R[>0]      \\
                                f  &\mapsto     \frac{(f : \phi)}{(f_0 : \phi)}
    \end{align*}
\end{thDef}
%
Der Titel der Definition deutet schon darauf hin, als was wir diese Funktionale
auffassen wollen: als Annäherungen an das gesuchte Haar-Integral. Dabei bekommen
wir einen besseren Wert, je kleiner der Träger von $\phi\in\contcompplus$ wird
(analog zur Idee in der Einleitung, $U$ auf $\{e\}$ zusammenschrumpfen zu
lassen). Dass diese $I_\phi$ schon die meisten gewünschten Eigenschaften
besitzen, zeigt folgendes Resultat:

\begin{thLemma}[Eigenschaften der approximierenden Funktionale]
    \label{pf:Iphiprops}
    %
    Seien $f,g,\phi \in\contcompplus$ und seien $x\in G$ sowie $\lambda\in\R[>0]$.
    Dann gelten für $I_\phi$ folgende Aussagen:
    \begin{enumerate}[a)]
        \item\label{pf:Iphiprops:leftinvariance}
            Linksinvarianz:\quad
            $I_\phi(L_xf) = I_\phi(f)$.
        \item\label{pf:Iphiprops:scalarmult}
            Verträglichkeit mit Skalarmultiplikation:\quad
            $I_\phi(\lambda f) = \lambda \, I_\phi(f)$.
        \item\label{pf:Iphiprops:subadditive}
            Subadditvität:\quad
            $I_\phi(f+g) \leq I_\phi(f) + I_\phi(g)$.
        \item\label{pf:Iphiprops:range}
            Eingrenzung des Wertebereichs:\quad
            $I_\phi(f) \in \bigl[ (f_0 : f)^{-1}, \; (f : f_0) \bigr]$.
    \end{enumerate}
\end{thLemma}

\begin{proof}
    Die ersten drei Eigenschaften folgen unmittelbar aus den entsprechenden 
    Aussagen in \cref{pf:covernumprops}.
    \\
    Wir zeigen noch \ref{pf:Iphiprops:range}: Betrachte die Definition von
    $I_\phi$ und verwende \mycref{pf:covernumprops:reducing}, um folgende
    Abschätzungen einzusehen:

    \begin{align*}
        \frac{(f : \phi)}{(f_0 : \phi)} 
        &\leq \frac{(f : f_0)\,(f_0 : \phi)}{(f_0 : \phi)}
        = (f : f_0)
        \\
        \shortintertext{und}
        %
        \frac{(f : \phi)}{(f_0 : \phi)} 
        &\geq \frac{(f : \phi)}{(f_0 : f)\,(f : \phi)}
        = (f_0 : f)^{-1}
        .
    \end{align*}
\end{proof}

Der einzige Defekt, den die $I_\phi$ gegenüber dem gesuchten Haar-Integral noch
haben, ist die fehlende Additivität. \mycref{pf:Iphiprops:subadditive} zeigt
zwar schon die Subadditvität auf, jedoch ist nun der entscheidende Punkt im
Beweis das folgende Lemma, welches zeigt, dass auch die umgekehrte Abschätzung
für jeden gegebenen Fehler für klein genuge Träger von $\phi$ erfüllt ist.
Etwas präziser formuliert bedeutet dies:

\begin{thLemma}[Nahezu-Additivität unter gewissen Bedinungen]
    Seien $f_1,f_2 \in\contcompplus$ und sei $\epsilon\in\R[>0]$ beliebig.
    Dann gibt es eine Umgebung $U\in\frakU$ um $e$ derart, dass für alle
    $\phi\in\contcompplus$ mit $\supp(\phi)\subset U$ gilt:
    \[ I_\phi(f_1) + I_\phi(f_2) \leq I_\phi(f_1+f_2) + \epsilon  \]
\end{thLemma}

\begin{proof}
    Sei $g\in\contcompplus$ derart, dass $g\vert_{\supp(f_1+f_2)} \equiv 1$
    gilt. Die Existenz eines solchen $g$ wird durch eine Variante des
    \emph{Urysohn'schen Lemmas} für lokalkompakten Hausdorfräume gesichert,
    wie man sie zum Beispiel bei Folland\cite[\S4,\;4.32]{bookc:folland99} 
    % TODO: Referenz auf vorherigen Vortrag einfügen
    findet. Weiter sei $\delta\in\R[>0]$ so gewählt, dass
    \[ 2\delta\, (f_1\mkern1mu{+}\mkern1mu f_2 : f_0) 
        + \delta(1+2\delta) \, (g : f_0) \leq \epsilon
    \]
    gilt. Warum wir $\delta$ gerade so wählen, ist zu diesem Zeitpunkt nicht
    offensichtlich und wird erst am Ende des Beweises klar werden.
    Wir definieren nun $h,h_1,h_2$ wie folgt:
    \[ h \defeq f_1+f_2+\delta g \qqundqq
        h_j \defeq \frac{f_j}{h} \quad \text{für $j\in\{1,2\}$,}
    \]
    wobei wir $h_1,h_2$ außerhalb von $\supp(f_1)$ bzw. $\supp(f_2)$ identisch
    null setzen. Dann gilt offenbar $h_1,h_2\in\contcompplus$, weswegen beide
    Funktionen nach \cref{tg:unicont} rechts-gleichmäßig stetig sind und es somit
    Umgebungen $U_1,U_2\in\frakU$ von $e$ gibt, so dass für $j\in\{1,2\}$ und
    alle $y\in U_j$ gilt: $\supnorm{h_j - R_yh_j} < \delta$. Setze dann $U\defeq
    U_1\cap U_2$. Sei weiter $\phi\in\contcompplus$ mit $\supp(\phi)\subset U$ und 
    $h\leq \sum_{i=1}^n c_i\,L_{x_i}\phi$ mit zugehörigen
    $n\in\N,\;c_i\in\R[>0],\; x_i\in G$ (wie in \cref{pf:def:covernum}).
    
    Sei ab jetzt $j\in\{1,2\}$.
    Ist dann $x\in G$ beliebig, so können wir abschätzen:
    \[  f_j(x) = h(x) \, h_j(x) \leq \sum_{i=1}^n c_i\,(L_{x_i}\phi)(x) \, h_j(x) \]
    Falls nun für $k\in\{1,\ldots,n\}$ das betrachtete $x$ in 
    $x_k \supp(\phi)$ liegt (sodass also $(L_{x_i}\phi)(x)\neq0$ gilt), 
    so folgt aus der Wahl von $\phi$ bzw. $U$:
    \[ \abs{ h_j(x_k) - h_j(x) } < \delta,  \qtextq{also insbesondere}
        h_j(x) < h_j(x_k) + \delta
    . \]
    Greifen wir damit die Abschätzung für $f_j$ wieder auf, so folgt:
    \[ 
        f_j(x) 
        \leq \sum_{i=1}^n c_i\,(L_{x_i}\phi)(x) \, \bigl( h_j(x_i) + \delta \bigr)
    \]
    Es hängt $h_j(x_i)$ nicht mehr von $x$ ab, womit wir daraus schließen dürfen:
    \[ (f_j : \phi) \leq \sum_{i=1}^n c_i \, \bigl( h_j(x_i) + \delta \bigr) \]
    %
    Wir addieren nun die letzte Ungleichung für $j=1$ und $j=2$ und erhalten:
    \[ (f_1 : \phi) + (f_2 : \phi) 
        \leq \sum_{i=1}^n c_i \, \bigl( 
        \underbrace{ h_1(x_i) + h_2(x_i) }_{
            \hspace*{10pt} \leq 1, \text{ nach Def. v. } h_1,h_2 \hspace*{-15pt} }
        + \;\; 2\delta \bigr)
        \leq \sum_{i=1}^n c_i \, (1+2\delta)
    \]
    Da $h\leq \sum_{i=1}^n c_i\,L_{x_i}\phi$ beliebig war, folgt daraus zunächst
    \[ (f_1 : \phi) + (f_2 : \phi)  \leq  (1+2\delta) \, (h : \phi)  \]
    und nach Multiplikation mit $(f_0 : \phi)^{-1}$ auch
    \[ I_\phi(f_1) + I_\phi(f_2)  \leq  (1+2\delta) \, I_\phi(h) . \]
    Nun erhalten wir aus $h=f_1+f_2+\delta g$ mit 
    \mycref{pf:Iphiprops:scalarmult} und \ref{pf:Iphiprops:subadditive}:
    \[ 
        I_\phi(f_1) + I_\phi(f_2) 
        \leq (1+2\delta) \, \bigl(
        I_\phi(f_1+f_2) + \delta I_\phi(g) 
        \bigr)
    \]
    Verwenden wir zusätzlich \mycref{pf:Iphiprops:range}, so bekommen wir auch
    noch die unerwünschten Abhängigkeiten von $\phi$ aus der Abschätzung
    entfernt und erhalten folgende Ungleichung:
    \begin{align*}
        I_\phi(f_1) + I_\phi(f_2) 
        &\leq I_\phi(f_1+f_2) + 2\delta \, (f_1\mkern1mu{+}\mkern1mu f_2 : f_0)
        + \delta(1+2\delta) \, (g : f_0)
        \\
        &\leq I_\phi(f_1+f_2) + \epsilon
    \end{align*}
    Die letzte Relation gilt dabei auf Grund der Wahl von $\delta$ zu Beginn,
    womit die Behauptung gezeigt ist.
    \\
\end{proof}





















\chapter{Anwendung: Die modulare Funktion}
Wir wollen nun eine erste Schlussfolgerung aus der soeben bewiesenen
Eindeutigkeit des Haar-Maßes ziehen. Sei dazu erneut $G$ eine LKH-Gruppe und sei
$\mu$ ein Haar-Maß auf $G$.

\begin{thLemma}%
    [Bildmaß eines linksinvarianten Maßes unter Rechtsmultiplikation]
    \label{mod:pushforwardmeasure}
    \hfill\\
    %
    Sei $y\in G$ und $\nu$ ein linksinvariantes Maß auf $G$.
    Dann ist auch das von der Rechtsmultiplikation mit $y^{-1}$ induzierte Maß
    $\nu \circ (x\mapsto xy)$ linksinvariant.
\end{thLemma}

\begin{proof}
    Nach dem Assiziativgesetz gilt für alle $B\in\borelsigmaalg(G)$ und $z\in G$:
    \[ \bigl(\nu \circ (x\mapsto xy)\bigr)(zB) 
        = \nu\bigl( (zB) y \bigr) 
        = \nu\bigl( z (By) \bigr)
        = \nu(By) = \bigl(\nu \circ (x\mapsto xy)\bigr)(B)
    \]
\end{proof}

\begin{thKorollar}%
    [Haar-Maße unter Rechtsmultiplikation]
    \label{mod:pushforwardhaar}
    %
    Für das Haar-Maß $\mu$ ist auch $\mu_y\colon B\mapsto\mu(By)$ ein Haar-Maß
    und es gibt ein $\Delta(y)\in\R[>0]$, so dass $\mu_y = \Delta(y)\,\mu$ gilt.
    Außerdem ist $\Delta(y)$ unabhängig von der Wahl von $\mu$.
\end{thKorollar}

\begin{proof}
    Die erste Aussage folgt unmittelbar aus \cref{mod:pushforwardmeasure} (und
    der Tatsache, dass $\mu_y$ auch von innen regulär ist, analog wie im Beweis
    von \cref{tg:rmeasuresVSfunctionals}) und
    die zweite aus \cref{pf:uniqueness}. Aus letzterem folgt außerdem die letzte
    Behauptung, denn: Ist $\nu$ ein weiteres Haar-Maß, so gilt $\mu = c\,\nu$ für
    ein $c\in\R[>0]$ und damit ergibt sich: 
    $\Delta(y)\,c\,\nu = \Delta(y)\,\mu = \mu_y = (c\,\nu)_y = c \, \nu_y$,
    woraus $\Delta(y)\,\nu = \nu_y$ folgt.
    \\
\end{proof}

Da wir mit \cref{mod:pushforwardhaar} gezeigt haben, dass $\Delta$ unabhängig
von der Wahl des Haar-Maßes ist, gibt dies Anlass zu folgender Definition:

\begin{thDef}[Modulare Funktion, unimodular]
    Die Abbildung $\Delta\colon G\to\R[>0]$ aus \cref{mod:pushforwardhaar}
    heißt \emph{modulare Funktion (von $G$)}.
    Gilt $\Delta \equiv 1$, so nennen wir $G$ \emph{unimodular}.
\end{thDef}

\begin{BspList}[\label{mod:unimodbsp}]
\item
    Jede abelsche LKH-Gruppe ist unimodular.
    
\item\label{mod:unimodbsp:kompakt}
    Ist die LKH-Gruppe $G$ kompakt, so ist sie unimodular. Begründung: Weil $G$
    kompakt ist, ist das Maß über den gesamten Raum endlich und es gilt dann: 
    \[ \infty > \Delta(y)\,\mu(G) = \mu(Gy) = \mu(G) . \]
    Wir dividieren durch $\mu(G)$ und erhalten $\Delta(y)=1$ für alle $y\in G$.
\end{BspList}

Zuletzt wollen wir noch zeigen, dass $\Delta\colon G \to \R[>0]$ sogar einige
schöne Eigenschaften besitzt, konkret:

\begin{thSatz}[Modulare Funktion als stetiger Gruppenhomomorphismus]
    Die modulare Funktion $\Delta$ ist ein stetiger Gruppenhomomorphismus
    von $G$ nach $(\R[>0],\,\cdot\,)$.
\end{thSatz}

\begin{proof}
    Seien $x,z\in G$ beliebig und sei $B\in\borelsigmaalg(G)$ kompakt. Dann gilt
    nach Definition der modularen Funktion:
    \[ \Delta(xz) \, \mu(B) = \mu(Bxz) = \mu\bigl( (Bx) z \bigr)
        = \Delta(z) \, \mu(Bx) = \Delta(z) \Delta(x) \mu(B)
    \]
    Da $\R[>0]$ abelsch ist, ist $\Delta$ also in der Tat ein Homomorphismus.
    Für den zweiten Teil gehen wir wie folgt vor: Wir zeigen die Stetigkeit von
    $\Delta$ im Punkt~$e$ und wie man sich leicht überlegt, folt aus der
    Stetigkeit eines Gruppenhomomorphismus im neutralen Element (oder einem
    beliebigen anderen Punkt) bereits die Stetigkeit auf der gesamten Gruppe.
    
    Zunächst nutzen wir aus, das nach dem Transformationssatz \pref{tg:trafo}
    für alle $y\in G$ und alle integrierbaren Funktionen $f$ auf $G$ gilt:
    \[ \int_G R_yf \dif\mu = \int_G f \dif{\mu_{y^{-1}}
        = \Delta(y^{-1}) \int_G f \dif\mu
    \]
    (Dabei sei $\mu_{y^{-1}}\colon B \mapsto \mu(By^{-1})$ wie in
    \cref{mod:pushforwardhaar}.)
    Wir können also auch zeigen, dass die linke Seite als Abbildung in~$y$
    stetig bei $e$ ist, denn dann muss dies für die rechte Seite und
    insbesondere für $\bigl(y\mapsto\Delta(y^{-1}})\bigr)\circ\bigl(y\mapsto
    y^{-1}\bigr) = \Delta$ auch gelten. Wähle dazu ein $f\in\contcompplus$ mit
    $\int_G f \dif\mu = 1$ und definiere
    \[ I\colon G\to\R[>0],\quad y\mapsto \int_G R_yf . \]
    Wegen $I(e)=1$ müssen wir also für jedes $\epsilon\in\R[>0]$ eine
    Umgebung $U\in\frakU$ um $e$ finden, so dass
    $I(U)\subset (1-\epsilon,1+\epsilon)$ oder äquivalent
    $\forall\,u\in U\colon \abs{I(u)-1} < \epsilon$ erfüllt ist.
    
    Wir setzen $K\defeq\supp(f)$ und wählen $\epsilon\in\R[>0]$ beliebig.
    Weil $G$ lokalkompakt ist, gibt es eine kompakte Umgebung $K'\in\frakU$ 
    um~$e$ und wir setzen $K''\defeq KK'$. Nach
    \mycref{tg:basics:KKcompact} ist dann $K''$ eine kompakte Menge und es gilt
    insbesondere $\mu(K'')<\infty$.
    Da $f$ nach \cref{tg:unicont} rechts-gleichmäßig stetig ist, existiert
    eine Umgebung $U'\in\frakU$ von~$e$, so dass für $y\in U'$ gilt:
    \[ \supnorm{f - R_yf} < \frac{\epsilon}{\mu(K'')}  \] 
    Setzte nun $U\defeq U'\cap K'$, was immer noch eine Umgebung um~$e$ ist.
    Dann gilt für alle $y\in U$:
    \[ \supp(f - R_yf) \subset K \cup KK' = K'' \]
    Nun können wir für $y\in U$ folgendermaßen abschätzen:
    \begin{align*}
        \abs{I(y)-1}                                            %
        &= \abs*{                                               %
            \int_G (R_yf - f) \dif\mu                           %
        }                                                       \\
        &\leq \int_G \abs{ f - R_yf } \dif\mu                   \\
        &\leq \frac{\epsilon}{\mu(K'')}\;\mu(K'') = \epsilon    %%
    \end{align*}
    Also ist $U$ die gesuchte Umgebung und wir sind fertig.
    \\
\end{proof}

Wenn man nun weiß, dass $\Delta$ ein stetiger Gruppenhomomorphismus ist, so kann
man für\ref{mod:unimodbsp:kompakt} von \cref{mod:unimodbsp} noch einen weiteren
schönen Beweis geben:\\
Ist $G$ kompakt, so muss $\Delta(G)\subset\R[>0]$ eine kompakte Untergruppe von
$(\R[>0],\,\cdot\,)$ sein. Wie man leicht zeigt, ist aber $\{1\}\leq\R[>0]$ die
einzige solche Untergruppe, womit $G$ schon unimodular sein muss.



















\nocite{bookc:folland95}
\nocite{www:mp:gruppenzwang7}

\appendix
\bibliographystyle{plaindin}
\bibliography{bibsources}

\end{document}






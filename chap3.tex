\chapter{Anwendung: Die modulare Funktion}
Wir wollen nun eine erste Schlussfolgerung aus der soeben bewiesenen
Eindeutigkeit des Haar-Maßes ziehen. Sei dazu erneut $G$ eine LKH-Gruppe und sei
$\mu$ ein Haar-Maß auf $G$.

\begin{thLemma}%
    [Bildmaß eines linksinvarianten Maßes unter Rechtsmultiplikation]
    \label{mod:pushforwardmeasure}
    \hfill\\
    %
    Sei $y\in G$ und $\nu$ ein linksinvariantes Maß auf $G$.
    Dann ist auch das von der Rechtsmultiplikation mit $y^{-1}$ induzierte Maß
    $\nu \circ (x\mapsto xy)$ linksinvariant.
\end{thLemma}

\begin{proof}
    Nach dem Assiziativgesetz gilt für alle $B\in\borelsigmaalg(G)$ und $z\in G$:
    \[ \bigl(\nu \circ (x\mapsto xy)\bigr)(zB) 
        = \nu\bigl( (zB) y \bigr) 
        = \nu\bigl( z (By) \bigr)
        = \nu(By) = \bigl(\nu \circ (x\mapsto xy)\bigr)(B)
    \]
\end{proof}

\begin{thKorollar}%
    [Haar-Maße unter Rechtsmultiplikation]
    \label{mod:pushforwardhaar}
    %
    Für das Haar-Maß $\mu$ ist auch $\mu_y\colon B\mapsto\mu(By)$ ein Haar-Maß
    und es gibt ein $\Delta(y)\in\R[>0]$, so dass $\mu_y = \Delta(y)\,\mu$ gilt.
    Außerdem ist $\Delta(y)$ unabhängig von der Wahl von $\mu$.
\end{thKorollar}

\begin{proof}
    Die erste Aussage folgt unmittelbar aus \cref{mod:pushforwardmeasure} (und
    der Tatsache, dass $\mu_y$ auch von innen regulär ist, analog wie im Beweis
    von \cref{tg:rmeasuresVSfunctionals}) und
    die zweite aus \cref{pf:uniqueness}. Aus letzterem folgt außerdem die letzte
    Behauptung, denn: Ist $\nu$ ein weiteres Haar-Maß, so gilt $\mu = c\,\nu$ für
    ein $c\in\R[>0]$ und damit ergibt sich: 
    $\Delta(y)\,c\,\nu = \Delta(y)\,\mu = \mu_y = (c\,\nu)_y = c \, \nu_y$,
    woraus $\Delta(y)\,\nu = \nu_y$ folgt.
    \\
\end{proof}

Da wir mit \cref{mod:pushforwardhaar} gezeigt haben, dass $\Delta$ unabhängig
von der Wahl des Haar-Maßes ist, gibt dies Anlass zu folgender Definition:

\begin{thDef}[Modulare Funktion, unimodular]
    Die Abbildung $\Delta\colon G\to\R[>0]$ aus \cref{mod:pushforwardhaar}
    heißt \emph{modulare Funktion (von $G$)}.
    Gilt $\Delta \equiv 1$, so nennen wir $G$ \emph{unimodular}.
\end{thDef}

\begin{BspList}[\label{mod:unimodbsp}]
\item
    Jede abelsche LKH-Gruppe ist unimodular.
    
\item\label{mod:unimodbsp:kompakt}
    Ist die LKH-Gruppe $G$ kompakt, so ist sie unimodular. Begründung: Weil $G$
    kompakt ist, ist das Maß über den gesamten Raum endlich und es gilt dann: 
    \[ \infty > \Delta(y)\,\mu(G) = \mu(Gy) = \mu(G) . \]
    Wir dividieren durch $\mu(G)$ und erhalten $\Delta(y)=1$ für alle $y\in G$.
\end{BspList}

Zuletzt wollen wir noch zeigen, dass $\Delta\colon G \to \R[>0]$ sogar einige
schöne Eigenschaften besitzt, konkret:

\begin{thSatz}[Modulare Funktion als stetiger Gruppenhomomorphismus]
    Die modulare Funktion $\Delta$ ist ein stetiger Gruppenhomomorphismus
    von $G$ nach $(\R[>0],\,\cdot\,)$.
\end{thSatz}

\begin{proof}
    Seien $x,z\in G$ beliebig und sei $B\in\borelsigmaalg(G)$ kompakt. Dann gilt
    nach Definition der modularen Funktion:
    \[ \Delta(xz) \, \mu(B) = \mu(Bxz) = \mu\bigl( (Bx) z \bigr)
        = \Delta(z) \, \mu(Bx) = \Delta(z) \Delta(x) \mu(B)
    \]
    Da $\R[>0]$ abelsch ist, ist $\Delta$ also in der Tat ein Homomorphismus.
    Für den zweiten Teil gehen wir wie folgt vor: Wir zeigen die Stetigkeit von
    $\Delta$ im Punkt~$e$ und wie man sich leicht überlegt, folt aus der
    Stetigkeit eines Gruppenhomomorphismus im neutralen Element (oder einem
    beliebigen anderen Punkt) bereits die Stetigkeit auf der gesamten Gruppe.
    
    Zunächst nutzen wir aus, das nach dem Transformationssatz \pref{tg:trafo}
    für alle $y\in G$ und alle integrierbaren Funktionen $f$ auf $G$ gilt:
    \[ \int_G R_yf \dif\mu = \int_G f \dif{\mu_{y^{-1}}
        = \Delta(y^{-1}) \int_G f \dif\mu
    \]
    (Dabei sei $\mu_{y^{-1}}\colon B \mapsto \mu(By^{-1})$ wie in
    \cref{mod:pushforwardhaar}.)
    Wir können also auch zeigen, dass die linke Seite als Abbildung in~$y$
    stetig bei $e$ ist, denn dann muss dies für die rechte Seite und
    insbesondere für $\bigl(y\mapsto\Delta(y^{-1}})\bigr)\circ\bigl(y\mapsto
    y^{-1}\bigr) = \Delta$ auch gelten. Wähle dazu ein $f\in\contcompplus$ mit
    $\int_G f \dif\mu = 1$ und definiere
    \[ I\colon G\to\R[>0],\quad y\mapsto \int_G R_yf . \]
    Wegen $I(e)=1$ müssen wir also für jedes $\epsilon\in\R[>0]$ eine
    Umgebung $U\in\frakU$ um $e$ finden, so dass
    $I(U)\subset (1-\epsilon,1+\epsilon)$ oder äquivalent
    $\forall\,u\in U\colon \abs{I(u)-1} < \epsilon$ erfüllt ist.
    
    Wir setzen $K\defeq\supp(f)$ und wählen $\epsilon\in\R[>0]$ beliebig.
    Weil $G$ lokalkompakt ist, gibt es eine kompakte Umgebung $K'\in\frakU$ 
    um~$e$ und wir setzen $K''\defeq KK'$. Nach
    \mycref{tg:basics:KKcompact} ist dann $K''$ eine kompakte Menge und es gilt
    insbesondere $\mu(K'')<\infty$.
    Da $f$ nach \cref{tg:unicont} rechts-gleichmäßig stetig ist, existiert
    eine Umgebung $U'\in\frakU$ von~$e$, so dass für $y\in U'$ gilt:
    \[ \supnorm{f - R_yf} < \frac{\epsilon}{\mu(K'')}  \] 
    Setzte nun $U\defeq U'\cap K'$, was immer noch eine Umgebung um~$e$ ist.
    Dann gilt für alle $y\in U$:
    \[ \supp(f - R_yf) \subset K \cup KK' = K'' \]
    Nun können wir für $y\in U$ folgendermaßen abschätzen:
    \begin{align*}
        \abs{I(y)-1}                                            %
        &= \abs*{                                               %
            \int_G (R_yf - f) \dif\mu                           %
        }                                                       \\
        &\leq \int_G \abs{ f - R_yf } \dif\mu                   \\
        &\leq \frac{\epsilon}{\mu(K'')}\;\mu(K'') = \epsilon    %%
    \end{align*}
    Also ist $U$ die gesuchte Umgebung und wir sind fertig.
    \\
\end{proof}

Wenn man nun weiß, dass $\Delta$ ein stetiger Gruppenhomomorphismus ist, so kann
man für\ref{mod:unimodbsp:kompakt} von \cref{mod:unimodbsp} noch einen weiteren
schönen Beweis geben:\\
Ist $G$ kompakt, so muss $\Delta(G)\subset\R[>0]$ eine kompakte Untergruppe von
$(\R[>0],\,\cdot\,)$ sein. Wie man leicht zeigt, ist aber $\{1\}\leq\R[>0]$ die
einzige solche Untergruppe, womit $G$ schon unimodular sein muss.


















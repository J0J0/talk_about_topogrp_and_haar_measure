
\chapter{Vorwort, Notation und Konventionen}
Dieses Skript behandelt \emph{Haar'sche Maße} auf \emph{topologischen Gruppen}.
Dazu wird zunächst der letztere Begriff eingeführt und anhand einiger Beispiele
verdeutlicht. Dann werden einige einfache Resultate gezeigt und schließlich der
Begriff des \emph{Haar-Maßes} eingeführt. Zuletzt wird die Existenz und
(im Wesentlichen) Eindeutigkeit des Haar-Maßes auf \emph{lokalkompakten
hausdorffschen topologischen Gruppen} gezeigt und als erste Anwendung die
\emph{modulare Funktion} besprochen.


\bigskip
In diesem Skript wird folgende Notation verwendet:
\begin{itemize}
    \item
        In Analogie zu der in englischer Literatur manchmal zu findenden
        Abkürzung \enquote{LCH space} werden wir für eine topologische Gruppe,
        die außerdem lokalkompakt und hausdorffsch ist, die Abkürzung
        \emph{LKH-Gruppe} verwenden.
        
    \item
        Sowohl $\subset$ als auch $\subseteq$ stehen für: enthalten oder gleich.
        Echt enthalten wird durch $\subsetneq$ gekennzeichnet.
    
    \item
        Die \emph{Natürlichen Zahlen $\N$} beginnen mit $0$.
    
    \item
        Zu einem Ring $R$ bezeichnet $R^\times$ die Einheitengruppe des Rings.
\end{itemize}


\bigskip
Weiterhin vereinbaren wir für einige Begriffe, die in der Literatur
unterschiedlich definiert werden, Folgendes% (wobei $(X,\Topo)$ ein beliebiger topologischer Raum sei)
:
\begin{itemize}
    \item
        Eine \emph{Umgebung} eines Punkts braucht nicht offen zu sein, sie muss
        nur eine eine \emph{offene Umgebung} des Punkts enthalten.
        
    \item 
        Ein topologischer Raum ist \emph{lokalkompakt} genau dann,
        wenn jeder Punkt eine kompakte Umgebung besitzt.
        
    \item
        Ein topologischer Raum ist \emph{regulär} genau dann,
        wenn er die Trennungsaxiome $T_2$ und $T_3$ erfüllt, d.\,h. wenn er ein
        Hausdorff-Raum ist und je eine (nicht-leere) abgeschlossene Menge und 
        ein Punkt aus dem Komplement selbiger durch (offene) Umgebungen
        getrennt werden können. (Man zeigt leicht, dass es schon genügt, 
        statt $T_2$ nur $T_0$ zu fordern, um eine äquivalente Definition zu
        erhalten.)
\end{itemize}
